\documentclass{amsproc}

\usepackage[utf8]{inputenc}
\usepackage[english]{babel}
\usepackage{comment}
\usepackage[alphabetic]{amsrefs}
\usepackage{tikz}
\usepackage{xcolor}
\usepackage{datetime2} 
\usepackage[colorlinks=true,linkcolor=blue,]{hyperref}

\usepackage{stackengine}
\usepackage{scalerel}
\usepackage{mathtools}
\usepackage{calc}
\usepackage{amsthm}
\usepackage{thmtools}
\usepackage[framemethod=TikZ]{mdframed}
\usepackage{amssymb}
\usepackage{amsfonts}
\usepackage{euscript}
% \def\EuScript#1{\mathscr#1}
% \usepackage{fourier-otf}
\usepackage{fourier}
%\usepackage{palatino}
% \usepackage[sc]{mathpazo} % add possibly `sc` and `osf` options
%\usepackage{eulervm}
%\usepackage{bbm}
%\usepackage{latexsym}
% \usepackage{mathrsfs}
% \usepackage{stmaryrd}
% \usepackage{stix}
\def\dotminus{\stackon[.2ex]{$-$}{$.$}}
% \def\wneg{\stackon[-.2ex]{$\neg$}{$\neg$}}
\usepackage{dsfont}
% \newcommand*{\TakeFourierOrnament}[1]{{%
% \fontencoding{U}\fontfamily{futs}\selectfont\char#1}}
% \renewcommand*{\danger}{\TakeFourierOrnament{66}}
\parindent0ex
\parskip1.2ex
% \makeatletter
% \DeclareOldFontCommand{\rm}{\normalfont\rmfamily}{\mathrm}
% \DeclareOldFontCommand{\sf}{\normalfont\sffamily}{\mathsf}
% \DeclareOldFontCommand{\tt}{\normalfont\ttfamily}{\mathtt}
% \DeclareOldFontCommand{\bf}{\normalfont\bfseries}{\mathbf}
% \DeclareOldFontCommand{\it}{\normalfont\itshape}{\mathit}
% \DeclareOldFontCommand{\sl}{\normalfont\slshape}{\@nomath\sl}
% \DeclareOldFontCommand{\sc}{\normalfont\scshape}{\@nomath\sc}
% \makeatother

\newcommand{\mylabel}[1]{{#1}\hfill}
\renewenvironment{itemize}
  {\begin{list}{$\triangleright$}{%
  \setlength{\parskip}{0mm}
  \setlength{\topsep}{.4\baselineskip}
  \setlength{\rightmargin}{0mm}
  \setlength{\listparindent}{0mm}
  \setlength{\itemindent}{0mm}
  \setlength{\labelwidth}{3ex}
  \setlength{\itemsep}{.2\baselineskip}
  \setlength{\parsep}{.2\baselineskip}
  \setlength{\partopsep}{0mm}
  \setlength{\labelsep}{1ex}
  \setlength{\leftmargin}{\labelwidth+\labelsep}
  \let\makelabel\mylabel}}{%
\end{list}}

\declaretheoremstyle[
  headfont=\normalfont\bfseries,
  notefont=\bfseries,
  notebraces={(}{)},
  bodyfont=\normalfont,
  postheadspace=1em,
  mdframed={
  nobreak=true,
  outerlinewidth=1pt,
  linecolor=gray!20,
  roundcorner = 1ex,    
  backgroundcolor=gray!10, 
  innerleftmargin=1ex,
  leftmargin=0ex,
  innerrightmargin=1ex,
  rightmargin=0ex,
  innertopmargin=1.5ex, 
  innerbottommargin=1ex, 
  skipabove=3ex,
  skipbelow=1ex}, 
]{mystyle}

\declaretheorem[style=mystyle]%
{theorem}
\declaretheorem[style=mystyle,sibling=theorem]%
{lemma,proposition,fact,corollary,definition,notation,remark,example,claim,question}

\let\proof\relax
\declaretheoremstyle[
  spaceabove=6pt, 
  spacebelow=6pt, 
  headfont=\normalfont\itshape, 
  bodyfont = \normalfont,
  postheadspace=1em, 
  qed=\qedsymbol, 
  headpunct={.}]
{myproof} 
\declaretheorem[style=myproof, unnumbered]{proof}

\renewcommand*{\emph}[1]{%
   \smash{\tikz[baseline]\node[rectangle, fill=teal!25, rounded corners, inner xsep=0.5ex, inner ysep=0.2ex, anchor=base, minimum height = 2.7ex]{\strut #1};}}

%%%%%%% GETCOMMIT
\newcommand\dotGitHEAD{}
\newcommand\branch{}


\makeatletter\let\myfilehandle\@inputcheck\makeatother

\openin\myfilehandle=.git/HEAD\relax

\begingroup\endlinechar-1
  \global\read\myfilehandle to \dotGitHEAD
\endgroup
\closein\myfilehandle

\newcommand\GetBranch{}
\def\GetBranch ref: refs/heads/#1\relax{\renewcommand{\branch}{#1}}

\expandafter\GetBranch\dotGitHEAD\relax

\makeatother

\linespread{1.1}
% \author{S. A. Polymath}
\author{Domenico Zambella}
\author{et al.}
\thanks{Dipartimento di Matematica, Universit\`a di Torino, via Carlo Alberto 10, 10123 Torino.}
\begin{document}
\title{Local stability in structures with a standard sort}
% \hfill\texttt{~}
\hfill\texttt{Branch:\ \branch\ \DTMnow}
\maketitle
\raggedbottom

\begin{abstract}

\end{abstract}


\def\ceq#1#2#3{\parbox[t]{18ex}{$\displaystyle #1$}\parbox{6ex}{\hfil $#2$}{$\displaystyle #3$}}

Let \emph{$S$\/} be some Hausdorff compact topological space.
We associate to $S$ a first order structure in a language \emph{${\EuScript L}_{\sf S}$\/} that has a symbol for each compact subsets $C\subseteq S^n$ and a function symbol for each continuous functions $f:S^n\to S$.
According to the context, $C$ and $f$ denote either the symbols of ${\EuScript L}_{\sf S}$ or their interpretation in the structure $S$.
Note that we write $x\in C$ for $C(x)$.

We also fix a first-order language \emph{${\EuScript L}_{\sf H}$\/} which we call the language of the \emph{home sort.}

\begin{definition}\label{def_0}
  Let \emph{${\EuScript L}$\/} be a two sorted language. 
  The two sorts are denoted by \emph{${\sf H}$} and \emph{${\sf S}$.} 
  The language ${\EuScript L}$ expands ${\EuScript L}_{\sf H}$ and ${\EuScript L}_{\sf S}$ with symbols sort ${\sf H}^n\times{\sf S}^m\to {\sf S}$.
  An \emph{${\EuScript L}$-structure\/} is a structure of signature ${\EuScript L}$ that interprets these symbols in equicontinuous functions.

  A \emph{standard structure\/} is a two-sorted ${\EuScript L}$-structure of the form $\langle M,S\rangle$, where $M$ is any structure of signature ${\EuScript L}_{\sf H}$ and $S$ is fixed.
  Standard structures are denoted by the domain of their home sort.
\end{definition}

We denote by ${\EuScript F}$ the set of ${\EuScript L}$-formulas constructed inductively from formulas of the form (i) ands (ii) below using Boolean connectives $\wedge$, $\vee$; the quantifiers $\forall\raisebox{1.1ex}{\scaleto{\sf H}{.8ex}\kern-.2ex}$, $\exists\raisebox{1.1ex}{\scaleto{\sf H}{.8ex}\kern-.2ex}$ of sort ${\sf H}$; and the quantifiers $\forall\raisebox{1.1ex}{\scaleto{\sf S}{.8ex}\kern-.2ex}$, $\exists\raisebox{1.1ex}{\scaleto{\sf S}{.8ex}\kern-.2ex}$ of sort ${\sf S}$

\begin{definition}
  We call atomic formulas those of the form
  \begin{itemize}
  \item[i.] atomic or negated atomic formulas of ${\EuScript L}_{\sf H}$;
  \item[ii.] $\tau\in C$, where $C\subseteq S$ is a compact set and $\tau$ is a term of sort ${\sf H}^n\times {\sf S}^m\to {\sf S}$.
  \end{itemize}
\end{definition}

In (ii) of the above definition we could replace $C$ by a compact subset of $S^n$ and $\tau$ with a tuple of terms~--~but this would complicate the notation and add nothing of relevance.

To convenienlty describe the substitution of a set/predicate $C\subseteq S$ occurring in a formula, we introduce variables of new sort ${\sf X}$.
Let ${\EuScript F}_{\sf X}$ be defined as ${\EuScript F}$ but replacing (ii) with 

\begin{itemize}
  \item[iii.] $\tau(x\,;\eta)\in X$, where $X$ is a variable of sort ${\sf X}$.
\end{itemize}

Formulas in ${\EuScript F}_{\sf X}$ are denoted by $\varphi(X)$, where $X=X_1,\dots,X_k$ be a tuple of variables of the new sort.
If $C=C _1,\dots,C_k$ is a tuple of compact subsets of $S$ then $\varphi(x\;\xi;C)$, is a formula in ${\EuScript F}$.
All formulas in ${\EuScript F}$ are obtained as such istantiations a formulas in ${\EuScript F}_{\sf X}$.

\begin{definition}
   The pseudonegation of $\varphi(X)$ is the formula obtained by replacing in $\varphi(X)$ the atomic formulas in  ${\EuScript L}_{\sf H}$ by their negation and each connective $\wedge$, $\vee$, $\forall\raisebox{1.1ex}{\scaleto{\sf H}{.8ex}\kern-.2ex}$, $\exists\raisebox{1.1ex}{\scaleto{\sf H}{.8ex}\kern-.2ex}$, $\forall\raisebox{1.1ex}{\scaleto{\sf S}{.8ex}\kern-.2ex}$, $\exists\raisebox{1.1ex}{\scaleto{\sf S}{.8ex}\kern-.2ex}$ by its respective dual  $\vee$, $\wedge$,  $\exists\raisebox{1.1ex}{\scaleto{\sf H}{.8ex}\kern-.2ex}$, $\forall\raisebox{1.1ex}{\scaleto{\sf H}{.8ex}\kern-.2ex}$, $\exists\raisebox{1.1ex}{\scaleto{\sf S}{.8ex}\kern-.2ex}$, $\forall\raisebox{1.1ex}{\scaleto{\sf S}{.8ex}\kern-.2ex}$.
   
  The pseudonegation of $\varphi(X)$ is denoted by $\tilde\varphi(X)$.
\end{definition}

Note that if $\varphi(X)$ is in ${\EuScript L}_{\sf H}$, so $\xi$ and $X$ are pleonastic, then $\tilde\varphi(X)\leftrightarrow\neg\varphi(X)$.
If $\tilde C$ is (componentwise) disjoint of $C$ then $\tilde\varphi(\tilde C)\leftarrow\neg\varphi(x\;\xi;C)$.

Let $\varphi(x\,;z\,;X)$ be a formula in  ${\EuScript F}_{\sf X}$ with the tuples of variables of sort ${\sf H}$ partitioned in two $x\,;z$.

A global $\varphi(x\,;z\,;X)$-type is a consistent set of formulas $p(x)\subseteq{\EuScript F}({\EuScript U})$ of the form $\varphi(x\,;b\,;C)$ and/or $\tilde\varphi(x\,;b\,;C)$ for some tuple $C$ of compact subsets of $S$ and some $b\in{\EuScript U}^{|z|}$.

Let $p(x)$ be a global $\varphi(x\,;z\,;X)$-type.
We say that the set ${\EuScript D}_p\subseteq {\EuScript U}^{|z|}\times S^{\scriptscriptstyle|X|}$ defined below is \emph{externally defined\/} by $p(x)$

\ceq{\hfill{\EuScript D}_{p}}{=}{\Big\{\langle b,\alpha\rangle\ :\  \varphi(x\,;b\,;C)\in p \textrm{ for every }C\textrm{ neighborhood of }\alpha\Big\}}

We say that ${\EuScript D}\subseteq {\EuScript U}^{|z|}\times S^{\scriptscriptstyle|X|}$ is \emph{approximated\/} by $\varphi(x\,;z\,;X)$ if for every finite $B\subseteq {\EuScript U}^{|z|}\times S^{\scriptscriptstyle|X|}$ there is a compact $C\subseteq S^{\scriptscriptstyle|X|}$ and some $a\in{\EuScript U}^{|x|}$ such that 


\ceq{\hfill \langle b,\alpha\rangle\in B\cap{\EuScript D}}{\Rightarrow}{\varphi(a\,;b\,;C) \textrm{ and } C \textrm{ is a neighborhood of }\alpha}

\ceq{\hfill \langle b,\alpha\rangle\in B\smallsetminus{\EuScript D}}{\Rightarrow}{\tilde\varphi(a\,;b\,;C) \textrm{ and } \neg\tilde C \textrm{ is a neighborhood of }\alpha }


\begin{fact}
  For every ${\EuScript D}\subseteq {\EuScript U}^{|z|}\times S^{\scriptscriptstyle|X|}$, the following are equivalent
  \begin{itemize}
    \item [1.] ${\EuScript D}$ is externally definable by some $\varphi(x\,;z\,;X)$-type
    \item [2.]  ${\EuScript D}$ is approximated by $\varphi(x\,;z\,;X)$.
  \end{itemize}
\end{fact}

\begin{proof}
  (1)$\Rightarrow$(2). Let $B\subseteq {\EuScript U}^{|z|}\times S^{\scriptscriptstyle|X|}$ be finite.
  For $\langle b,\alpha\rangle\in B\cap{\EuScript D}$ let $C_{\langle b,\alpha\rangle}$ be some neightborhood of $\alpha$. % such that $\varphi(x\,;b\,;C_{\langle b,\alpha\rangle})\in p$.
  Let $C$ be the union of all these neighborhoods.
  For $\langle b,\alpha\rangle\in B\smallsetminus{\EuScript D}$ let $\tilde C_{\langle b,\alpha\rangle}$ be some neightborhood of $\alpha$. % such that $\tilde\varphi(x\,;b\,;\tilde C_{\langle b,\alpha\rangle})\in p$.
  Let $\tilde C$ be the union of all these neighborhoods.
  By the finiteness of $B$, we can require that $C$ and $\tilde C$ are disjoint. 
  Then (2) follows from the concistency of $p(x)$.

  (2)$\Rightarrow$(1). Note that the condition of approximability asserts the finite concistency of the type $p(x)$ that is union of the following two sets of formulas

\noindent\kern5ex $\Big\{\varphi(x\,;b\,;C)\ :\ \langle b,\alpha\rangle\in{\EuScript D}\textrm{ and }C\textrm{ neighborhood of }\alpha\Big\}$


\noindent\kern5ex $\Big\{\tilde\varphi(x\,;b\,;\tilde C)\ :\ \langle b,\alpha\rangle\notin{\EuScript D}\textrm{ and }\tilde C\textrm{ neighborhood of }\alpha\Big\}$

The inclusion ${\EuScript D}\subseteq{\EuScript D}_p$ is immediate.
For the converse inclusion, let $\langle b,\alpha\rangle\notin{\EuScript D}$ and let $\tilde C$ be a neighborhood of $\alpha$ such that $\tilde\varphi(x\,;b\,;\tilde C)$.
\end{proof}

%%%%%%%%%%%%%%%%%%%%%%%%%%%%%%%%%%%
%%%%%%%%%%%%%%%%%%%%%%%%%%%%%%%%%%%
%%%%%%%%%%%%%%%%%%%%%%%%%%%%%%%%%%%
%%%%%%%%%%%%%%%%%%%%%%%%%%%%%%%%%%%
%%%%%%%%%%%%%%%%%%%%%%%%%%%%%%%%%%%
\section{}

\def\ceq#1#2#3{\parbox[t]{23ex}{$\displaystyle #1$}\parbox{6ex}{\hfil $#2$}{$\displaystyle #3$}}

%%%%%%%%%%%%%%%%%%%%%
%%%%%%%%%%%%%%%%%%%%%
%%%%%%%%%%%%%%%%%%%%%
%%%%%%%%%%%%%%%%%%%%%
%%%%%%%%%%%%%%%%%%%%%
%%%%%%%%%%%%%%%%%%%%%
%%%%%%%%%%%%%%%%%%%%%
\newcommand\biburl[1]{\url{#1}}
\BibSpec{arXiv}{%
  +{}{\PrintAuthors}{author}
  +{,}{ \textit}{title}
  +{}{ \parenthesize}{date}
  +{,}{ arXiv:}{eprint}
  +{,}{ } {note}
  % +{,}{ \url}
}

\BibSpec{webpage}{%
  +{}{\PrintAuthors} {author}
  +{,}{ \textit} {title}
  +{,}{ } {portal}
  +{}{ \parenthesize} {date}
  +{,}{ } {doi}
  +{,}{ } {note}
  +{.}{ } {transition}
}
\begin{bibdiv}
\begin{biblist}[]\normalsize

\bib{clcl}{article}{
    label={ABBMZ},
    author = {Agostini, Claudio},
    author = {Baratella, Stefano},
    author = {Barbina, Silvia},
    author = {Motto Ros, Luca},
    author = {Zambella, Domenico},
    title = {Continuous logic in a classical setting},
    note={t.a. in Bull. Iranian Math. Soc.},
    date = {2025},
  }

  \bib{A}{webpage}{
    label={A},
    author={Auslander, Josef},
    title={Topological Dynamics},
    portal={Scholarpedia},
    doi={doi:10.4249/ scholarpedia.3449},
    date={2008},
}

  \bib{BY}{article}{
    label={BY},
    author={Ben Yaacov, Ita\"{\i}},
    title={Model theoretic stability and definability of types, after A.
    Grothendieck},
    journal={Bull. Symb. Log.},
    volume={20},
    date={2014},
    number={4},
    pages={491--496},
    % issn={1079-8986},
    % review={\MR{3294276}},
    % doi={10.1017/bsl.2014.33},
 }

\bib{BBHU}{article}{
  label={BBHU},
  author={Ben Yaacov, Ita\"{\i}},
  author={Berenstein, Alexander},
  author={Henson, C. Ward},
  author={Usvyatsov, Alexander},
  title={Model theory for metric structures},
  conference={
      title={Model theory with applications to algebra and analysis. Vol. 2},
  },
  book={
      series={London Math. Soc. Lecture Note Ser.},
      volume={350},
      publisher={Cambridge Univ. Press, Cambridge},
  },
  %  isbn={978-0-521-70908-8},
  date={2008},
  pages={315--427},
  %  review={\MR{2436146}},
  %  doi={10.1017/CBO9780511735219.011},
}

% \bib{G}{article}{
%   label={G},
%   author = {Goldbring, Isaac},
%   title = {Lecture notes on nonstandard analysis},
%   conference={
%     title={UCLA Summer School in Logic 2012}
%    },
%   eprint = {www.math.uci.edu/~isaac/NSA\%20notes.pdf},
% }

% \bib{H}{arXiv}{
%   label={H},
%   author = {Hart, Bradd},
%   title = {An Introduction To Continuous Model Theory},
%   eprint={2303.03969},
%   doi = {10.48550/arXiv.2303.03969},
%   url = {https://arxiv.org/abs/},
%   publisher = {arXiv},
%   date = {2023},
% }

\bib{Hr}{article}{
   label={Hr},
   author={Hrushovski, Ehud},
   title={Stable group theory and approximate subgroups},
   journal={J. Amer. Math. Soc.},
   volume={25},
   date={2012},
   number={1},
   pages={189--243},
   %issn={0894-0347},
   %doi={10.1090/S0894-0347-2011-00708-S},
}

% \bib{HPP}{article}{
%    label={HPP},
%    author={Hrushovski, Ehud},
%    author={Peterzil, Ya'acov},
%    author={Pillay, Anand},
%    title={Groups, measures, and the NIP},
%    journal={J. Amer. Math. Soc.},
%    volume={21},
%    date={2008},
%    number={2},
%    pages={563--596},
%   %  issn={0894-0347},
%   %  review={\MR{2373360}},
%   %  doi={10.1090/S0894-0347-07-00558-9},
% }

\bib{HI}{article}{
  label={HI},
  author={Henson, C. Ward},
  author={Iovino, Jos\'{e}},
  title={Ultraproducts in analysis},
  conference={
    title={Analysis and logic},
    address={Mons},
    date={1997},
   },
   book={
      series={London Math. Soc. Lecture Note Ser.},
      volume={262},
      publisher={Cambridge Univ. Press, Cambridge},
   },
   date={2002},
   pages={1--110},
  %  review={\MR{1967834}},
}

\bib{K}{article}{
  label={K},
  author = {Keisler, H. Jerome},
  title = {Model Theory for Real-valued Structures},
  book={
    editor={Iovino, Jos\'e},
    title ={{in} Beyond First Order Model Theory}, 
    volume ={II},
    publisher={Chapman and Hall/CRC},
  % https://doi.org/10.1201/9780429263637
    date = {2023},
}
}
% \bib{Z}{arXiv}{
%   label = {Z},
%   author = {Zambella, Domenco},
%   note = {In preparation},
% }

% \bib{Z}{article}{
%   label={Z},
%    author={Zambella, Domenico},
%    title={Elementary classes of finite VC-dimension},
%    journal={Arch. Math. Logic},
%    volume={54},
%    date={2015},
%    number={5-6},
%    pages={511--520},
%   %  issn={0933-5846},
%   %  review={\MR{3372605}},
%   %  doi={10.1007/s00153-015-0424-0},
% }



%%%%%%%%%%%%%%%%%%%%%%%%%%%%%%%%%%%
%%%%%%%%%%%%%%%%%%%%%%%%%%%%%%%%%%%
%%%%%%%%%%%%%%%%%%%%%%%%%%%%%%%%%%%
%%%%%%%%%%%%%%%%%%%%%%%%%%%%%%%%%%%
%%%%%%%%%%%%%%%%%%%%%%%%%%%%%%%%%%%
%%%%%%%%%%%%%%%%%%%%%%%%%%%%%%%%%%%
% \section{Continuous group actions}
\end{biblist}
\end{bibdiv}
\end{document}