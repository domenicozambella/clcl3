\documentclass{amsproc}

\usepackage[utf8]{inputenc}
\usepackage[english]{babel}
\usepackage{comment}
\usepackage[alphabetic]{amsrefs}
\usepackage{tikz}
\usepackage{xcolor}
\usepackage{datetime2} 
\usepackage[colorlinks=true,linkcolor=blue,]{hyperref}
\usepackage{tcolorbox}

\usepackage{stackengine}
\usepackage{scalerel}
\usepackage{mathtools}
\usepackage{calc}
\usepackage{amsthm}
\usepackage{thmtools}
\usepackage[framemethod=TikZ]{mdframed}
\usepackage{amssymb}
\usepackage{amsfonts}
\usepackage{euscript}
% \def\EuScript#1{\mathscr#1}
% \usepackage{fourier-otf}
\usepackage{fourier}
%\usepackage{palatino}
% \usepackage[sc]{mathpazo} % add possibly `sc` and `osf` options
%\usepackage{eulervm}
%\usepackage{bbm}
%\usepackage{latexsym}
% \usepackage{mathrsfs}
% \usepackage{stmaryrd}
% \usepackage{stix}
\def\dotminus{\stackon[.2ex]{$-$}{$.$}}
% \def\wneg{\stackon[-.2ex]{$\neg$}{$\neg$}}
\usepackage{dsfont}
% \newcommand*{\TakeFourierOrnament}[1]{{%
% \fontencoding{U}\fontfamily{futs}\selectfont\char#1}}
% \renewcommand*{\danger}{\TakeFourierOrnament{66}}
\parindent0ex
\parskip1.2ex
% \makeatletter
% \DeclareOldFontCommand{\rm}{\normalfont\rmfamily}{\mathrm}
% \DeclareOldFontCommand{\sf}{\normalfont\sffamily}{\mathsf}
% \DeclareOldFontCommand{\tt}{\normalfont\ttfamily}{\mathtt}
% \DeclareOldFontCommand{\bf}{\normalfont\bfseries}{\mathbf}
% \DeclareOldFontCommand{\it}{\normalfont\itshape}{\mathit}
% \DeclareOldFontCommand{\sl}{\normalfont\slshape}{\@nomath\sl}
% \DeclareOldFontCommand{\sc}{\normalfont\scshape}{\@nomath\sc}
% \makeatother

\newcommand{\mylabel}[1]{{#1}\hfill}
\renewenvironment{itemize}
  {\begin{list}{$\triangleright$}{%
  \setlength{\parskip}{0mm}
  \setlength{\topsep}{.4\baselineskip}
  \setlength{\rightmargin}{0mm}
  \setlength{\listparindent}{0mm}
  \setlength{\itemindent}{0mm}
  \setlength{\labelwidth}{3ex}
  \setlength{\itemsep}{.2\baselineskip}
  \setlength{\parsep}{.2\baselineskip}
  \setlength{\partopsep}{0mm}
  \setlength{\labelsep}{1ex}
  \setlength{\leftmargin}{\labelwidth+\labelsep}
  \let\makelabel\mylabel}}{%
\end{list}}


\newtheoremstyle{mio}% name
     {2\parskip}     % Space above
     {2\parskip}     % Space below
     {}%         Body font
     {}%         Indent amount (empty = no indent, \parindent = para indent)
     {\bfseries}% Thm head font
     {}%        Punctuation after thm head
     {1ex}%     Space after thm head: " " = normal interword space;
           %   \newline = linebreak
     {\llap{\thmnumber{#2}\hskip0.9ex}\thmname{#1}\thmnote{\bfseries{}#3}}% Thm head spec (can be left empty, meaning `normal')

\newcounter{thm}

\tcbset{
  mythm/.style={
    colback=black!5!white,
    colframe=white!50!black,
    extrude right by=2ex,
    extrude left by=6.5ex,
    % before=\par\noindent,
    % after=\par\noindent,
    top=0.7ex,
    bottom=1.5ex,
    right=2ex,
    left=6.5ex,
    boxsep=0pt,
    % parbox=false,
    before skip=\baselineskip,
    after skip=\baselineskip,
  },
}
\theoremstyle{mio}
\newtheorem{theorem}[thm]{Theorem}\tcolorboxenvironment{theorem}{mythm}
\newtheorem{corollary}[thm]{Corollary}\tcolorboxenvironment{corollary}{mythm}
\newtheorem{proposition}[thm]{Proposition}\tcolorboxenvironment{proposition}{mythm}
\newtheorem{lemma}[thm]{Lemma}\tcolorboxenvironment{lemma}{mythm}
\newtheorem{fact}[thm]{Fact}\tcolorboxenvironment{fact}{mythm}
\newtheorem{definition}[thm]{Definition}\tcolorboxenvironment{definition}{mythm}
\newtheorem{assumption}[thm]{Assumption}\tcolorboxenvironment{assumption}{mythm}
\newtheorem{void}[thm]{}\tcolorboxenvironment{void}{mythm}
\newtheorem{notation}[thm]{Notation}\tcolorboxenvironment{notation}{mythm}
\newtheorem{note}[thm]{Note}\tcolorboxenvironment{note}{mythm}
\newtheorem{remark}[thm]{Remark}\tcolorboxenvironment{remark}{mythm}
\newtheorem{exercise}[thm]{Exercise}
\newtheorem{example}[thm]{Example}
\newtheorem{question}[thm]{Question}

\makeatletter
\providecommand{\proofNameStyle}{\bfseries}
\renewenvironment{proof}[1][\proofname]{\par
  \pushQED{\qed}%
  \normalfont% \topsep6\p@\@plus6\p@\relax
  % \vspace*{-\baselineskip}
  \trivlist
  \item[\hskip\labelsep
        \proofNameStyle
    #1\@addpunct{.}]\ignorespaces
}{%
  \popQED\endtrivlist\@endpefalse
}
\makeatother



\renewcommand*{\emph}[1]{%
   \smash{\tikz[baseline]\node[rectangle, fill=teal!25, rounded corners, inner xsep=0.5ex, inner ysep=0.2ex, anchor=base, minimum height = 2.7ex]{\strut #1};}}

%%%%%%% GETCOMMIT
\newcommand\dotGitHEAD{}
\newcommand\branch{}


\makeatletter\let\myfilehandle\@inputcheck\makeatother

\openin\myfilehandle=.git/HEAD\relax

\begingroup\endlinechar-1
  \global\read\myfilehandle to \dotGitHEAD
\endgroup
\closein\myfilehandle

\newcommand\GetBranch{}
\def\GetBranch ref: refs/heads/#1\relax{\renewcommand{\branch}{#1}}

\expandafter\GetBranch\dotGitHEAD\relax

\makeatother

\linespread{1.1}
\author{L. S. S. S. S. Polymath}
% \author{Domenico Zambella}
% \author{al.}
\thanks{Dipartimento di Matematica, Universit\`a di Torino, via Carlo Alberto 10, 10123 Torino.}
\begin{document}
\title{Local stability in structures with a standard sort}
% \hfill\texttt{~}
\hfill\texttt{Branch:\ \branch\ \DTMnow}
\maketitle
\raggedbottom

\begin{abstract}

\end{abstract}

\def\medrel#1{\parbox{5ex}{\hfil $#1$}}
\def\ceq#1#2#3{\parbox[t]{18ex}{$\displaystyle #1$}\medrel{#2}{$\displaystyle #3$}}

\section{Structures with a standard sort}

Let \emph{$S$\/} be some Hausdorff compact topological space.
We associate to $S$ a first order structure in a language \emph{${\EuScript L}_{\sf S}$\/} that has a symbol for each compact subset $C\subseteq S$ and a function symbol for each continuous functions $f:S^n\to S$.
According to the context, $C$ and $f$ denote either the symbols of ${\EuScript L}_{\sf S}$ or their interpretation in the structure $S$.
We write \emph{$x\in C$\/} for $C(x)$.

Finally, note that we could allow in ${\EuScript L}_{\sf S}$ relation symbols for all compact subsets of $S^n$, for any $n$.
But this would clutter the notation adding very little to the theory.

\begin{definition}\label{def_0}
  Let \emph{${\EuScript L}$\/} be a two sorted language. 
  The two sorts are denoted by \emph{${\sf H}$} and \emph{${\sf S}$.} 
  The language ${\EuScript L}$ expands ${\EuScript L}_{\sf H}$ and ${\EuScript L}_{\sf S}$ with symbols sort ${\sf H}^n\times{\sf S}^m\to {\sf S}$.
  An \emph{${\EuScript L}$-structure\/} is a structure of signature ${\EuScript L}$ that interprets these symbols in equicontinuous functions.

  A \emph{standard structure\/} is a two-sorted ${\EuScript L}$-structure of the form $\langle M,S\rangle$, where $M$ is any structure of signature ${\EuScript L}_{\sf H}$ and $S$ is fixed.
  Standard structures are denoted by the domain of their home sort.
\end{definition}

We denote by ${\EuScript F}$ the set of ${\EuScript L}$-formulas constructed inductively from atomic formulas of the form (i) and (ii) below using Boolean connectives $\wedge$, $\vee$; the quantifiers $\forall\raisebox{1.1ex}{\scaleto{\sf H}{.8ex}\kern-.2ex}$, $\exists\raisebox{1.1ex}{\scaleto{\sf H}{.8ex}\kern-.2ex}$ of sort ${\sf H}$; and the quantifiers $\forall\raisebox{1.1ex}{\scaleto{\sf S}{.8ex}\kern-.2ex}$, $\exists\raisebox{1.1ex}{\scaleto{\sf S}{.8ex}\kern-.2ex}$ of sort ${\sf S}$.

\begin{definition}
  We call atomic formulas those of the form
  \begin{itemize}
  \item[i.] atomic or negated atomic formulas of ${\EuScript L}_{\sf H}$;
  \item[ii.] $\tau\in C$, where $C\subseteq S$ is a compact set and $\tau$ is a term of sort ${\sf H}^n\times {\sf S}^m\to {\sf S}$.
  \end{itemize}
\end{definition}

\begin{remark}
  A smaller fragment of ${\EuScript F}$ may be of interest in some specific contexts.
  This is obtained by excluding equalities and inequalities from (i) and requiring that $C$ in (ii) is a regular compact set.
  The discussion below applies to this smaller fragment as well.
\end{remark}

 Let $M\subseteq N$ be standard structures.
 We say that $M$ is an \emph{${\EuScript F}$-elementary\/} substructure of $N$ if the latter models all sentences in ${\EuScript F}(M)$ true in $M$.

 A standard structure $M$ is \emph{${\EuScript F}$-saturated\/} if it realizes every type $p(x\,;\xi)\subseteq{\EuScript F}(A)$, for any $A\subseteq M$ of cardinality smaller than $|M|$, that is finitely consistent in $M$.
 The following is proved in \cite{clcl} for signatures containing only symbols sort ${\sf H}^n\times{\sf S}^m\to {\sf S}$ with $n{\cdot}m=0$.
 In \cite{Z} it is observed that, under the assumption of equicontinuity, the proof extends to the case $n{\cdot}m\geq 0$.
\begin{theorem}[ (Compactness)]%
  Every standard structure has a ${\EuScript F}$-saturated ${\EuScript F}$-elementary extension.
\end{theorem}

To conveniently describe classes of ${\EuScript F}$ that only differ by the sets/predicates $C\subseteq S$ that occur in the formula, we introduce variables of new sort ${\sf X}$.
These will be used as placeholders for compact subsets of $S$.
Let ${\EuScript F}_{\sf X}$ be defined as ${\EuScript F}$ but replacing (ii) with

\begin{itemize}
  \item[iii.] $\tau(x\,;\xi)\in X$, where $X$ is a variable of sort ${\sf X}$.
\end{itemize}

Formulas in ${\EuScript F}_{\sf X}$ are denoted by $\varphi(x\,;\xi;X)$, where $X=X_1,\dots,X_k$ be a tuple of variables of sort ${\sf X}$.
If $C=C _1,\dots,C_k$ is a tuple of compact subsets of $S$ then $\varphi(x\,;\xi;C)$, is a formula in ${\EuScript F}$.
This we call an \emph{instance\/} of $\varphi(x\,;\xi;X)$.
All formulas in ${\EuScript F}$ are instances of formulas in ${\EuScript F}_{\sf X}$.

\begin{definition}
   Let $\varphi\in{\EuScript F}_{\sf X}$ be formula~--~possibly with some (hidden) free variables.
   The \emph{pseudonegation\/} of $\varphi\in{\EuScript F}_{\sf X}$ is the formula obtained by replacing in $\varphi$ the atomic formulas in  ${\EuScript L}_{\sf H}$ by their negation and each connective $\wedge$, $\vee$, $\forall\raisebox{1.1ex}{\scaleto{\sf H}{.8ex}\kern-.2ex}$, $\exists\raisebox{1.1ex}{\scaleto{\sf H}{.8ex}\kern-.2ex}$, $\forall\raisebox{1.1ex}{\scaleto{\sf S}{.8ex}\kern-.2ex}$, $\exists\raisebox{1.1ex}{\scaleto{\sf S}{.8ex}\kern-.2ex}$ by its respective dual  $\vee$, $\wedge$,  $\exists\raisebox{1.1ex}{\scaleto{\sf H}{.8ex}\kern-.2ex}$, $\forall\raisebox{1.1ex}{\scaleto{\sf H}{.8ex}\kern-.2ex}$, $\exists\raisebox{1.1ex}{\scaleto{\sf S}{.8ex}\kern-.2ex}$, $\forall\raisebox{1.1ex}{\scaleto{\sf S}{.8ex}\kern-.2ex}$.
   
  The pseudonegation of $\varphi$ is denoted by \emph{${\sim}\varphi$.}
  Clearly, when $X$ does not occur in $\varphi$, we have ${\sim}\varphi\leftrightarrow\neg\varphi$.
\end{definition}


The following fact will be used without further mention.

\begin{fact}
  Let $\varphi(X)\in{\EuScript F}_{\sf X}$ have, possibly, hidden variables of sort ${\sf H}$ and/or ${\sf S}$.
  Then for every (tuples of) compact sets $C\subseteq C'$ and $\tilde C\cap C=\varnothing$.
  \smallskip

  \ceq{\hfill\varphi(C)}{\rightarrow}{\phantom{\neg}\varphi(C')}\smallskip

  \ceq{\hfill{\sim}\varphi(C)}{\rightarrow}{\neg\varphi(\tilde C)}
\end{fact}

A standard structure $M$ is \emph{${\EuScript F}$-maximal\/} if it models all ${\EuScript F}(M)$-sentences that hold in some of its ${\EuScript F}$-elementary extensions.

\begin{fact}\label{fact_maximal}
  Let $\varphi(X)\in{\EuScript F}_{\sf X}(M)$.
  Then the following are equivalent
  \begin{itemize}
    \item [1.] $M$ is ${\EuScript F}$-maximal
    \item [2.] for every tuple $C$ of compact subsets of $S$
    
    \ceq{\hfill M\ \models\quad\varphi(C)}{\leftrightarrow}{\bigwedge\Big\{\varphi(C')\ :\ C'\textrm{ neighborhood of }C\Big\} }
  
    \item [3.] for every tuple $C$ of compact subsets of $S$

    \ceq{\hfill M\ \models\quad \neg\varphi(C)}{\leftrightarrow}{\bigvee\Big\{{\sim}\varphi(\tilde C)\ :\ \tilde C\cap C=\varnothing\Big\}.}
  \end{itemize}
  Moreover, all of the above hold when $M$ is ${\EuScript F}$-saturated.
\end{fact}

\begin{proof}
  Arguing by induction on the syntax of $\varphi(X)$ we can prove that (2) holds when $M$ is ${\EuScript F}$-saturated, see \cite{clcl}.

  2$\Leftrightarrow$3. By normality.

  1$\Rightarrow$2. If (2) fails, then $\neg\varphi(C)\wedge\neg\,{\sim}\varphi(\tilde C)$ holds in $M$ for every $\tilde C\cap C=\varnothing$.
  Let ${\EuScript U}$ be an ${\EuScript F}$-saturated ${\EuScript F}$-elementary extension of $M$.
  As $M$ is ${\EuScript F}$-maximal, $\neg\varphi(C)\wedge\neg\,{\sim}\varphi(\tilde C)$ also holds in ${\EuScript U}$.
  Then (2) fails also in ${\EuScript U}$, a contradiction.

  2$\Rightarrow$1. 
  Let $N$ be any ${\EuScript F}$-elementary extension of $M$.
  Suppose $N\models\varphi(C)$.
  By (2), it suffices to prove that $M\models\varphi(C')$ for every $C'$ neighborhood of $C$.
  Suppose not. Then, by (3), $M\models{\sim}\varphi(\tilde C)$ for some $\tilde C\cap C'=\varnothing$.
  Then $N\models{\sim}\varphi(\tilde C)$ by ${\EuScript F}$-elementarity.
  As $\tilde C\cap C=\varnothing$, this contradicts the above fact.
\end{proof}


%%%%%%%%%%%%%%%%%%%%%%%%%%%%%%%
%%%%%%%%%%%%%%%%%%%%%%%%%%%%%%%
%%%%%%%%%%%%%%%%%%%%%%%%%%%%%%%
\section{Externally definable sets}

In what follows we fix a large saturated standard structure which we denote by \emph{${\EuScript U}$.}
Unless otherwise specified, we work inside ${\EuScript U}$.

Let $\varphi(x\,;z\,;X)$ be a formula in  ${\EuScript F}_{\sf X}$ with the variables of sort ${\sf H}$ partitioned in two tuples $x\,;z$.
A \emph{global $\varphi(x\,;z\,;X)$-type\/} is a maximally consistent set of formulas that are instances of $\varphi(x\,;b\,;X)$ and/or ${\sim}\varphi(x\,;b\,;X)$ for some $b\in{\EuScript U}^{|z|}$.

% \begin{fact}
%   Let $p(x)$ be a global $\varphi(x\,;z\,;X)$-type.
% \end{fact}

Let $p(x)$ be a global $\varphi(x\,;z\,;X)$-type.
We define 

\ceq{\hfill\emph{${\EuScript D}_{p,\varphi,C}$}}{=}{\big\{b\in{\EuScript U}^{z}\ :\ \varphi(x\,;b\,;C)\in p\big\}}

\begin{fact}\ \vskip-0.5\baselineskip
  
  \ceq{\hfill\neg{\EuScript D}_{p,\varphi,C}}{=}{\bigcup\Big\{{\EuScript D}_{p,{\sim}\varphi,\tilde C}\ \ :\ \ \tilde C\cap C=\varnothing\Big\}}
\end{fact}

\begin{proof}
  ($\subseteq$).
  Let $b\notin{\EuScript D}_{p,\varphi,C}$.
  Let $A\subseteq{\EuScript U}$ be such that every instance of $\varphi(x\,;b\,;X)$ and ${\sim}\varphi(x\,;b\,;X)$ that is consistent with $p(x)\restriction A$ is finitely consistent with $p(x)$.
  Finally pick $a\models p(x)\restriction A$.
  As $\neg\varphi(a,b,C)$, from Fact~\ref{fact_maximal} we obtain that  ${\sim}\varphi(a\,;b\,;\tilde C)$ holds for some $\tilde C\cap C=\varnothing$.
  Then ${\sim}\varphi(x\,;b\,;\tilde C)$ is consistent with $p(x)\restriction A$.
  By maximality of $p(x)$, we have ${\sim}\varphi(x\,;b\,;\tilde C)\in p$.
  Then $b\in{\EuScript D}_{p,{\sim}\varphi,\tilde C}$.
  
  ($\supseteq$).
  Similar.
\end{proof}

\begin{fact}
  Let $\varphi(x\,;z\,;X)$ and $C$ be given.
  Then for every ${\EuScript D}\subseteq {\EuScript U}^{z}$ the following are equivalent
  \begin{itemize}
    \item [1.] ${\EuScript D}={\EuScript D}_{p,\varphi,C}$ for some global $\varphi(x\,;z\,;X)$-type $p(x)$
    \item [2.] for every finite set $B\subseteq{\EuScript U}^{z}$ there is an $a\in{\EuScript U}^{x}$ such that
    
    \ceq{\hfill B\ \cap\ {\EuScript D}}{=}{B\ \cap\ \varphi(a\,;{\EuScript U}^x\,;C).}

  \end{itemize}
\end{fact}

\begin{proof}
  Let $B_0=B\smallsetminus{\EuScript D}$ and $B_1=B\cap{\EuScript D}$.

  1$\Rightarrow$2. 
  By the above fact, there is a $\tilde C\cap C=\varnothing$ such that $b\in{\EuScript D}_{p,\sim\varphi,\tilde C}$  form every $b\in B_0$.
  Let $a$ realize the type containing $\varphi(x\,;b\,;C)$ for every $b\in B_1$ and ${\sim}\varphi(x\,;b\,;\tilde C)$ for every $b\in B_0$. 

  2$\Rightarrow$1. 
  There is a set $\tilde C\cap C=\varnothing$ such that ${\sim}\varphi(x\,;b\,;\tilde C)$  for every $b\in B_0$.
  Let $p(x)$ be any global $\varphi(x\,;z\,;X)$-type containing $\varphi(x\,;b\,;C)$ for every $b\in B_1$ and ${\sim}\varphi(x\,;b\,;\tilde C)$ for every $b\in B_0$.
  Such type exists as (2) guarantees consistency.
\end{proof}

Let ${\EuScript D}\subseteq {\EuScript U}^{z}\times S^{\scriptscriptstyle|X|}$.
For $\alpha\in S^{\scriptscriptstyle|X|}$, we write ${\EuScript D}_\alpha$ for the set of all $b\in{\EuScript U}^{z}$ such that $\langle b,\alpha\rangle\in{\EuScript D}$.
We say that ${\EuScript D}$ is \emph{externally definable\/} by $\varphi(x\,;z\,;X)$ if there is a global $\varphi(x\,;z\,;X)$-type $p(x)$ such that for every $\alpha$

\ceq{\hfill{\EuScript D}_\alpha}{=}{\bigcap\big\{{\EuScript D}_{p,\varphi,C}\ :\ \alpha\in C^\circ\big\}}

We say that ${\EuScript D}$ is \emph{approximable\/} by $\varphi(x\,;z\,;X)$ if for every $\alpha\in S^{\scriptscriptstyle|X|}$ and every finite $B\subseteq{\EuScript U}^{z}$ there is an $a\in{\EuScript U}^{|x|}$ such that

\ceq{\hfill B\ \cap\ {\EuScript D}_\alpha}{=}{B\ \cap\ \bigcap\big\{\varphi(a\,;{\EuScript U}^{z}\,;C)\ :\ \alpha\in C^\circ\big\}}

\begin{fact}
  Let $\varphi(x\,;z\,;X)$ be given.
  Then, for every ${\EuScript D}\subseteq {\EuScript U}^{z}\times S^{\scriptscriptstyle|X|}$ the following are equivalent
  \begin{itemize}
    \item [1.] ${\EuScript D}$ is externally definable by $\varphi(x\,;z\,;X)$
    \item [2.] ${\EuScript D}$ is approximated by $\varphi(x\,;z\,;X)$.
  \end{itemize}
\end{fact}

We say that ${\EuScript D}$ is approximable \emph{from below\/} if for every $B\subseteq{\EuScript D}$ there is
an $a\in{\EuScript U}^{|x|}$ such that

\ceq{\hfill B}{\subseteq}{\varphi(a\,;{\EuScript U}^z\,;C)}\medrel{\subseteq}${\EuScript D}$.

%%%%%%%%%%%%%%%%%%%%%%%%%
%%%%%%%%%%%%%%%%%%%%%%%%%%
%%%%%%%%%%%%%%%%%%%%%%%%%%
%%%%%%%%%%%%%%%%%%%%%%%%%%
%%%%%%%%%%%%%%%%%%%%%%%%%%
\section{Stable formulas}

We say that $\varphi(x\,;z\,;X)$ is unstable if there are some compact sets $C\cap\tilde C=\varnothing$ and a sequence $\langle a_i\,;b_i\ :\ i<\omega\rangle$ such that for every $i<j<\omega$

\ceq{\hfill \varphi(a_i\,;b_j\,;C)}{\wedge}{{\sim}\varphi(a_j\,;b_i\,;\tilde C)}

We say that $\varphi(x\,;z\,;X)$ is \emph{stable\/} if it is not unstable.

% \begin{theorem}[ (conjecture)]\label{thm_stable}
%   If $\varphi(x\,;z\,;X)$ is stable, then for every $\varphi(x\,;z\,;X)$-type $p(x)$ and every compact set $C$ then\smallskip

%   \ceq{\hfill{\EuScript D}_{p,\varphi,C}}{=}{\bigcup_{i=1}^{n}\ \bigcap_{j=1}^{n}\ \varphi(a_{i,j}\,;{\EuScript U}^z\,;C)}\smallskip

%    for some $\langle a_{i,j}\ :\ 1\le i,j\le n\rangle$.
% \end{theorem}

The following fact is a characterization of stability in a context of particular interest.
Let $\tau:{\EuScript U}^{x;z}\to S$ be a definable function.
We say that $\tau(x\,;z)$ is \emph{stable\/} if so is the formula $\tau(x\,;z)\in X$.

\def\medrel#1{\parbox{5ex}{\hfil $#1$}}
\def\ceq#1#2#3{\parbox[t]{25ex}{$\displaystyle #1$}\medrel{#2}{$\displaystyle #3$}}

\begin{fact}
  Let $\tau:{\EuScript U}^{x;z}\to S$ be a definable function.
  Then the following are equivalent
  \begin{itemize}
    \item [1.] the formula $\tau(x\,;z)\in X$ is unstable
    \item [2.] for some sequence $\langle a_i\,;b_i\ :\ i<\omega\rangle$\smallskip

    \ceq{\hfill \lim_{i\to\infty}\lim_{j\to\infty}\tau(a_i\,;b_j)}{\neq}{\lim_{j\to\infty}\lim_{i\to\infty}\tau(a_i\,;b_j)}

    \item [3.] for some sequence $\langle a_i\,;b_i\ :\ i<\omega\rangle$ and some neighborhood $\varepsilon$ of the diagonal of $S^2$\smallskip

    \ceq{\hfill \big\langle\tau(a_i\,;b_j),\ \tau(a_j\,;b_i)\big\rangle}{\notin}{\varepsilon}\smallskip

    for every $i<j<\omega$.
  \end{itemize}
\end{fact}

\begin{proof}
  2$\Leftrightarrow$3. Clear.

  1$\Rightarrow$3.
  Let $C\cap\tilde C=\varnothing$ be such that $\tau(a_i\,;b_j)\in C$ and $\tau(a_j\,;b_i)\in \tilde C$ for every $i<j<\omega$ as given by (1).
  Then (3) holds for any $\varepsilon$ contained in the complement of $C\times\tilde C$.%\ \cup\ \tilde C\times C

  2$\Rightarrow$1. Let $C$ and $\tilde C$ be two disjoint intervals centered in the two limits in (2).
  Then (1) is witnessed by a tail of the sequence $\langle a_i\,;b_i\ :\ i<\omega\rangle$.
\end{proof}

\begin{fact}
  Let $S=[0,1]$.
  Let $\tau(x\,;z)$ be a stable definable function.  
  Let $f:{\EuScript U}^{x;z}\to S$.
  Assume that for every finite $B\subseteq{\EuScript U}^z$ and every $\varepsilon>0$ there is an $a\in{\EuScript U}^x$ such that\smallskip

  \ceq{\hfill \big|f(b)-\tau(a\,;b)\big|}{\le}{\varepsilon}\hfill for every $b\in B$.\smallskip

  Then there are some $\langle a_{i,j}\ :\ 1\le i,j\le n\rangle$ such that\smallskip

  \ceq{\hfill f(b)}{=}{\max_{i=1,\dots,n}\ \min_{j=1,\dots,n}\ \tau(a_{i,j}\,;b)}\hfill for every $b\in{\EuScript U}^z$.

\end{fact}

\begin{proof}
  
\end{proof}

\begin{lemma}
  Let $\varphi(x\,;z\,;X)$ be stable.
  Let ${\EuScript D}\subseteq {\EuScript U}^{|z|}$ be approximated by $\varphi(x\,;z\,;X)$ from below.
  Then
  
  \ceq{\hfill{\EuScript D}_{p,\varphi,C}}{=}{\bigcup_{i=1}^{n}\ \varphi(a_i\,;{\EuScript U}^z\,;C)}\smallskip

  for some $\langle a_i\ :\ 1\le i\le n\rangle$.
\end{lemma}

\vskip3ex
Garbage after this line
\vskip-\baselineskip
\hrulefill
\vskip3ex

We define ${\EuScript D}_p\subseteq {\EuScript U}^{|z|}\times S^{\scriptscriptstyle|X|}$ as follows

\ceq{\hfill{\EuScript D}_{p}}{=}{\Big\{\,\langle b,\alpha\rangle\ :\  \varphi(x\,;b\,;C)\in p\ \textrm{ for every }C\textrm{ neighborhood of }\alpha\Big\}.}

% \ceq{\hfill\tilde {\EuScript D}_{p}}{=}{\Big\{\langle b,\alpha\rangle\ :\  {\sim}\varphi(x\,;b\,;\tilde C)\in p \textrm{ for every }C\rotatebox[origin=c]{180}{$\notin$}\alpha\Big\}.}

\begin{fact}\ \vskip-0.5\baselineskip

  \ceq{\hfill\neg {\EuScript D}_{p}}{=}{\Big\{\,\langle b,\alpha\rangle\ :\  {\sim}\varphi(x\,;b\,;\tilde C)\in p\ \textrm{  for every }\tilde C\,\rotatebox[origin=c]{180}{$\notin$}\,\alpha\;\Big\}.}
\end{fact}

\begin{proof}
  Let $a\models p(x)\restriction b$.
  Let $C$ be the intersection of all neighborhoods of $\alpha$ such that $\varphi(a\,;b\,;C')$.
  Assume $\langle b,\alpha\rangle\notin{\EuScript D}_p$, in other words $\varphi(a\,;b\,;C)$, then $C$ is a neighborhood of $\alpha$.
  By Fact~\ref{fact_maximal} we have $\varphi(a\,;b\,;C)$.
  Let $\tilde C\subseteq C\smallsetminus\{\alpha\}$ be such that $C$ is a neighborhood of $\tilde C$.
  Then $\neg\varphi(a\,;b\,;\tilde C)$.
  Therefor by the definition of ${\EuScript D}_p$.

  Then we can find $\tilde C$ such that $\alpha\notin\tilde C$ and $C$ is a neighborhood of $\tilde C$.
  contains two dijoint compacts
  Then ${\sim}\varphi(x\,;b\,;\tilde C')\in p$ by the definition of ${\EuScript D}_p$.


  Then ${\sim}\varphi(x\,;b\,;\tilde C')\in p$ for every neighborhood $\tilde C'$ of $\alpha$ disjoint from $C'$.
  Then for every neighborhood $C$ of $\alpha$, we have $\varphi(x\,;b\,;C)\in p$.
\end{proof}

We say that ${\EuScript D}$ is \emph{externally defined\/} by $p(x)$ and $\varphi(x\,;z\,;X)$ if ${\EuScript D}_p\subseteq{\EuScript D}$ and $\tilde{\EuScript D}_p\subseteq\neg{\EuScript D}$.

We say that ${\EuScript D}\subseteq {\EuScript U}^{|z|}\times S^{\scriptscriptstyle|X|}$ is \emph{approximated\/} by $\varphi(x\,;z\,;X)$ if for every finite $B\subseteq {\EuScript U}^{|z|}\times S^{\scriptscriptstyle|X|}$ there is an $a\in{\EuScript U}^{|x|}$ such that 

\ceq{\hfill \langle b,\alpha\rangle\in B\cap{\EuScript D}}{\Rightarrow}{\varphi(a\,;b\,;C) \textrm{ for every neighborhood } C \textrm{ of }\alpha}

\ceq{\hfill \langle b,\alpha\rangle\in B\smallsetminus{\EuScript D}}{\Rightarrow}{{\sim}\varphi(a\,;b\,;C) \textrm{ for every }C\textrm{ such that }\alpha\notin C}

\begin{fact}
  For every ${\EuScript D}\subseteq {\EuScript U}^{|z|}\times S^{\scriptscriptstyle|X|}$, the following are equivalent
  \begin{itemize}
    \item [1.] ${\EuScript D}$ is externally definable by some global $\varphi$-type
    \item [2.]  ${\EuScript D}$ is approximated by $\varphi(x\,;z\,;X)$.
  \end{itemize}
\end{fact}

\begin{proof}
  (1)$\Rightarrow$(2). Let $B\subseteq {\EuScript U}^{|z|}\times S^{\scriptscriptstyle|X|}$ be finite.
  For $\langle b,\alpha\rangle\in B\cap{\EuScript D}$ let $C_{\langle b,\alpha\rangle}$ be some neightborhood of $\alpha$. % such that $\varphi(x\,;b\,;C_{\langle b,\alpha\rangle})\in p$.
  Let $C$ be the union of all these neighborhoods.
  For $\langle b,\alpha\rangle\in B\smallsetminus{\EuScript D}$ let $\tilde C_{\langle b,\alpha\rangle}$ be some neightborhood of $\alpha$. % such that ${\sim}\varphi(x\,;b\,;\tilde C_{\langle b,\alpha\rangle})\in p$.
  Let $\tilde C$ be the union of all these neighborhoods.
  By the finiteness of $B$, we can require that $C$ and $\tilde C$ are disjoint. 
  Then (2) follows from the concistency of $p(x)$.

  (2)$\Rightarrow$(1). Note that the condition of approximability asserts the finite concistency of the type $p(x)$ that is union of the following two sets of formulas

\noindent\kern5ex $\Big\{\varphi(x\,;b\,;C)\ :\ \langle b,\alpha\rangle\in{\EuScript D}\textrm{ and }C\textrm{ neighborhood of }\alpha\Big\}$


\noindent\kern5ex $\Big\{{\sim}\varphi(x\,;b\,;C)\ :\ \langle b,\alpha\rangle\notin{\EuScript D}\textrm{ and }\alpha\notin C\Big\}$

The inclusion ${\EuScript D}\subseteq{\EuScript D}_p$ is immediate.
For the converse inclusion, let $\langle b,\alpha\rangle\notin{\EuScript D}$ and let $\tilde C$ be a neighborhood of $\alpha$ such that ${\sim}\varphi(x\,;b\,;\tilde C)$.
\end{proof}

%%%%%%%%%%%%%%%%%%%%%%%%%%%%%%%%%%%
%%%%%%%%%%%%%%%%%%%%%%%%%%%%%%%%%%%
%%%%%%%%%%%%%%%%%%%%%%%%%%%%%%%%%%%
%%%%%%%%%%%%%%%%%%%%%%%%%%%%%%%%%%%
%%%%%%%%%%%%%%%%%%%%%%%%%%%%%%%%%%%
\section{}

\def\ceq#1#2#3{\parbox[t]{23ex}{$\displaystyle #1$}\parbox{6ex}{\hfil $#2$}{$\displaystyle #3$}}

%%%%%%%%%%%%%%%%%%%%%
%%%%%%%%%%%%%%%%%%%%%
%%%%%%%%%%%%%%%%%%%%%
%%%%%%%%%%%%%%%%%%%%%
%%%%%%%%%%%%%%%%%%%%%
%%%%%%%%%%%%%%%%%%%%%
%%%%%%%%%%%%%%%%%%%%%
\newcommand\biburl[1]{\url{#1}}
\BibSpec{arXiv}{%
  +{}{\PrintAuthors}{author}
  +{,}{ \textit}{title}
  +{}{ \parenthesize}{date}
  +{,}{ arXiv:}{eprint}
  +{,}{ } {note}
  % +{,}{ \url}
}

\BibSpec{webpage}{%
  +{}{\PrintAuthors} {author}
  +{,}{ \textit} {title}
  +{,}{ } {portal}
  +{}{ \parenthesize} {date}
  +{,}{ } {doi}
  +{,}{ } {note}
  +{.}{ } {transition}
}
\begin{bibdiv}
\begin{biblist}[]\normalsize

\bib{clcl}{article}{
    label={AAVV},
    author = {Agostini, Claudio},
    author = {Baratella, Stefano},
    author = {Barbina, Silvia},
    author = {Motto Ros, Luca},
    author = {Zambella, Domenico},
    title = {Continuous logic in a classical setting},
    note={t.a. in Bull. Iranian Math. Soc.},
    date = {2025},
  }

  \bib{A}{webpage}{
    label={A},
    author={Auslander, Josef},
    title={Topological Dynamics},
    portal={Scholarpedia},
    doi={doi:10.4249/ scholarpedia.3449},
    date={2008},
}

  \bib{BY}{article}{
    label={BY},
    author={Ben Yaacov, Ita\"{\i}},
    title={Model theoretic stability and definability of types, after A.
    Grothendieck},
    journal={Bull. Symb. Log.},
    volume={20},
    date={2014},
    number={4},
    pages={491--496},
    % issn={1079-8986},
    % review={\MR{3294276}},
    % doi={10.1017/bsl.2014.33},
 }

\bib{BBHU}{article}{
  label={BBHU},
  author={Ben Yaacov, Ita\"{\i}},
  author={Berenstein, Alexander},
  author={Henson, C. Ward},
  author={Usvyatsov, Alexander},
  title={Model theory for metric structures},
  conference={
      title={Model theory with applications to algebra and analysis. Vol. 2},
  },
  book={
      series={London Math. Soc. Lecture Note Ser.},
      volume={350},
      publisher={Cambridge Univ. Press, Cambridge},
  },
  %  isbn={978-0-521-70908-8},
  date={2008},
  pages={315--427},
  %  review={\MR{2436146}},
  %  doi={10.1017/CBO9780511735219.011},
}

% \bib{G}{article}{
%   label={G},
%   author = {Goldbring, Isaac},
%   title = {Lecture notes on nonstandard analysis},
%   conference={
%     title={UCLA Summer School in Logic 2012}
%    },
%   eprint = {www.math.uci.edu/~isaac/NSA\%20notes.pdf},
% }

% \bib{H}{arXiv}{
%   label={H},
%   author = {Hart, Bradd},
%   title = {An Introduction To Continuous Model Theory},
%   eprint={2303.03969},
%   doi = {10.48550/arXiv.2303.03969},
%   url = {https://arxiv.org/abs/},
%   publisher = {arXiv},
%   date = {2023},
% }


% \bib{Hr}{article}{
%    label={Hr},
%    author={Hrushovski, Ehud},
%    title={Stable group theory and approximate subgroups},
%    journal={J. Amer. Math. Soc.},
%    volume={25},
%    date={2012},
%    number={1},
%    pages={189--243},
%    %issn={0894-0347},
%    %doi={10.1090/S0894-0347-2011-00708-S},
% }

% \bib{HPP}{article}{
%    label={HPP},
%    author={Hrushovski, Ehud},
%    author={Peterzil, Ya'acov},
%    author={Pillay, Anand},
%    title={Groups, measures, and the NIP},
%    journal={J. Amer. Math. Soc.},
%    volume={21},
%    date={2008},
%    number={2},
%    pages={563--596},
%   %  issn={0894-0347},
%   %  review={\MR{2373360}},
%   %  doi={10.1090/S0894-0347-07-00558-9},
% }

\bib{HI}{article}{
  label={HI},
  author={Henson, C. Ward},
  author={Iovino, Jos\'{e}},
  title={Ultraproducts in analysis},
  conference={
    title={Analysis and logic},
    address={Mons},
    date={1997},
   },
   book={
      series={London Math. Soc. Lecture Note Ser.},
      volume={262},
      publisher={Cambridge Univ. Press, Cambridge},
   },
   date={2002},
   pages={1--110},
  %  review={\MR{1967834}},
}

\bib{K}{article}{
  label={K},
  author = {Keisler, H. Jerome},
  title = {Model Theory for Real-valued Structures},
  book={
    editor={Iovino, Jos\'e},
    title ={{in} Beyond First Order Model Theory}, 
    volume ={II},
    publisher={Chapman and Hall/CRC},
  % https://doi.org/10.1201/9780429263637
    date = {2023},
}
}
% \bib{Z}{arXiv}{
%   label = {Z},
%   author = {Zambella, Domenco},
%   note = {In preparation},
% }

% \bib{Z}{article}{
%   label={Z},
%    author={Zambella, Domenico},
%    title={Elementary classes of finite VC-dimension},
%    journal={Arch. Math. Logic},
%    volume={54},
%    date={2015},
%    number={5-6},
%    pages={511--520},
%   %  issn={0933-5846},
%   %  review={\MR{3372605}},
%   %  doi={10.1007/s00153-015-0424-0},
% }

\bib{Z}{arXiv}{
  label={Z},
  author = {Zambella, Domenico},
  title = {Standed analysis},
  eprint={2311.15711},
  doi = {10.48550/arXiv.2311.15711},
  url = {https://arxiv.org/abs/},
  publisher = {arXiv},
  date = {2023},
}

%%%%%%%%%%%%%%%%%%%%%%%%%%%%%%%%%%%
%%%%%%%%%%%%%%%%%%%%%%%%%%%%%%%%%%%
%%%%%%%%%%%%%%%%%%%%%%%%%%%%%%%%%%%
%%%%%%%%%%%%%%%%%%%%%%%%%%%%%%%%%%%
%%%%%%%%%%%%%%%%%%%%%%%%%%%%%%%%%%%
%%%%%%%%%%%%%%%%%%%%%%%%%%%%%%%%%%%
% \section{Continuous group actions}
\end{biblist}
\end{bibdiv}
\end{document}