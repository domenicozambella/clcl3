% !TEX root = clcl3.tex
\section{Stable definable functions}
\def\medrel#1{\parbox{5ex}{\hfil $#1$}}
\def\ceq#1#2#3{\parbox[t]{33ex}{$\displaystyle #1$}\medrel{#2}{$\displaystyle #3$}}

In this section we specialize the notions introduced in the previous sections to formulas $\varphi(x\,;z\,;X)$ of the form $\tau(x\,;z)\in X$ where $\tau(x\,;z)$ is an $S$-valued ${\EuScript F}$-definable function.
Notice that ${\sim}\big(\tau(x\,;z)\in X\big)=\tau(x\,;z)\in X$.

We say that $\tau(x\,;z)$ is \emph{stable\/} if so is $\tau(x\,;z)\in X$.
The following fact is a suggestive characterization of the stability of functions.

\begin{fact}
  Let $\tau(x\,;z)$ be as above.
  Then the following are equivalent
  \begin{itemize}
    \item [1.] the formula $\tau(x\,;z)\in X$ is unstable
    \item [2.] there is a sequence $\langle a_i\,;b_i\ :\ i<\omega\rangle$ such that\smallskip
    
    \noindent\kern-\labelwidth\kern-\labelsep
    \ceq{\hfill \lim_{i\to\infty}\lim_{j\to\infty}\tau(a_i\,;b_j)}{\neq}{\lim_{j\to\infty}\lim_{i\to\infty}\tau(a_i\,;b_j)}

    \item [3.] there is a sequence $\langle a_i\,;b_i\ :\ i<\omega\rangle$ and some $\varepsilon$ such that for every $i<j<\omega$\smallskip
    
    \noindent\kern-\labelwidth\kern-\labelsep
    \ceq{\hfill\big\langle\tau(a_i\,;b_j),\ \tau(a_j\,;b_i)\big\rangle}{\notin}{\varepsilon.}

    % \item [3.] for some sequence $\langle a_i\,;b_i\ :\ i<\omega\rangle$ and some neighborhood $\varepsilon$ of the diagonal of $S^2$\smallskip

    % \ceq{\hfill \big\langle\tau(a_i\,;b_j),\ \tau(a_j\,;b_i)\big\rangle}{\notin}{\varepsilon}\smallskip

    % for every $i<j<\omega$.
  \end{itemize}
\end{fact}

\begin{proof}
  2$\Leftrightarrow$3 Clear.

  1$\Rightarrow$2.
  Let $C\cap\tilde C=\varnothing$ and $\langle a_i\,;b_i\ :\ i<\omega\rangle$ be as given by (1).
  That is, $\tau(a_i\,;b_j)\in C$ and $\tau(a_j\,;b_i)\in \tilde C$ hold for every $i<j<\omega$.
  We can restrict to a subsequence such that the two limits exist; $C$ contains the limit on the left; and $\tilde C$ contains the limit on the right~--~which therefore are distinct.%\ \cup\ \tilde C\times C

  2$\Rightarrow$1. Let $C$ and $\tilde C$ be disjoint neighborhoods of the two limits in (2).
  Then (1) is witnessed by a tail of the sequence $\langle a_i\,;b_i\ :\ i<\omega\rangle$.
\end{proof}

% Let ${\EuScript D}\subseteq{\EuScript U}^{z}\times K(S)$.

Let $f:{\EuScript U}^{z}\to S$ be a function.
We define
{\def\medrel#1{\parbox{5ex}{\hfil $#1$}}
\def\ceq#1#2#3{\parbox[t]{15ex}{$\displaystyle #1$}\medrel{#2}{$\displaystyle #3$}}

\ceq{\hfill{\EuScript D\!}_f}{=}{\big\{\langle b,C\rangle\ \ :\ \ f(b)\in C,\ \ b\in{\EuScript U}^{z},\ \ C\in K(S)\big\}}}

% Note that ${\sim}{\EuScript D\!}_f={\EuScript D\!}_f$ therefore, by Fact~\ref{fact_~definibile} and~\ref{fact_~inclusione} the implications in Definition~\ref{def_approx_X} are in fact equivalences.

\begin{fact}
  The following are equivalent
  \begin{itemize}
    \item [1] ${\EuScript D\!}_f$ is approximable by $\tau(x\,;z)\in X$
    \item [2] for every finite $B\subseteq{\EuScript U}^z$ there is an $a\in{\EuScript U}^x$ such that $\tau(a\,;b)=f(b)$ for every $b\in B$.\smallskip
  \end{itemize}
\end{fact}

\begin{fact}[ (conjecture)]
  The following are equivalent
  \begin{itemize}
    \item [1] ${\EuScript D\!}_f$ is $\delta$-approximable by $\tau(x\,;z)\in X$
    \item [2] for every finite $B\subseteq{\EuScript U}^z$ there is an $a\in{\EuScript U}^x$ such that $\big\langle\tau(a\,;b)\,;f(b)\big\rangle\in\delta$ for every $b\in B$.
  \end{itemize}
\end{fact}
