% !TEX root = clcl3.tex
\section{An approximate version of approximability}

We present a second version of Theorem~\ref{thm_epsilon_stability_definability} which uses a weaker variant of $\delta$-approx\-imability.
This notion applies naturally to definable functions which we discuss in the next section.

\begin{definition}\label{def_approx_X}\strut
  We say that ${\EuScript D}$ is \emph{$\delta$-approximable\/} by $\varphi(x\,;z\,;X)$ if for every finite $B\subseteq{\EuScript U}^z$ there is an $a\in{\EuScript U}^x$ such that\smallskip

  \ceq{1.\hfill \langle b,C\rangle\in{\EuScript D}}{\Rightarrow}{\varphi(a\,;b\,;C^\delta)}\quad and\smallskip

  \ceq{2.\hfill\langle b,C^\delta\rangle\in {\EuScript D}}{\Leftarrow}{\varphi(a\,;b\,;C)}\hfill for every $b\in B$ and every $C$.\smallskip

   We say  $\delta$-approximable from \emph{from $\varepsilon$-below\/} if we can also require that\smallskip

  \ceq{2'.\hfill\langle b,C^{\delta+\varepsilon}\rangle\in {\EuScript D}}{\Leftarrow}{\varphi(a\,;b\,;C)}\hfill for every $b\in{\EuScript U}$ and every $C$.

\end{definition}
It is evident that, when ${\EuScript D}$ is closed as defined in Section~\ref{K(S)} then approximable is equivalent to $\delta$-approximable for every $\delta$.

 We restate a variant of Theorem~\ref{thm_epsilon_stability_definability}.
 The proof follows closely that in the previous section.

\begin{theorem}\label{thm_epsilon_delta_stability_definability}
  Let $\varphi(x\,;z\,;X)$ be stable.
  Assume that ${\EuScript D}$ is $\delta$-approximable by $\varphi(x\,;z\,;X)$.
  Then for every $\varepsilon$ there are some $\langle a_{i,j}\ :\ i< k,\ j<m\rangle$ such that for every $b\in{\EuScript U}^z$ and every $C$\medskip

  \ceq{\hfill \langle b,C\rangle\in{\EuScript D}}{\Rightarrow}{\bigvee_{i< k}\ \bigwedge_{j<m}\ \varphi(a_{i,j}\,;b\,;C^{\delta+\varepsilon})}\medskip

  \ceq{\hfill \langle b,C^{\delta+\varepsilon}\rangle\in{\EuScript D}}{\Leftarrow}{\bigvee_{i< k}\ \bigwedge_{j<m}\ \varphi(a_{i,j}\,;b\,;C)}
\end{theorem}

\begin{proof}
  The theorem is an immediate consequence of the following two lemmas and Lemma~\ref{lem_sigma_stable} in the previous section.
\end{proof}



\begin{lemma}
  Let $\varphi(x\,;z\,;X)$ be stable.
  Assume that ${\EuScript D}$ is $\delta$-approximable by $\varphi(x\,;z\,;X)$ from $\varepsilon$-below.
  Then there is are some $\langle a_i\ :\ i<k\rangle$ such that for every $b\in{\EuScript U}^z$ and every $C$\medskip

  \ceq{1.\hfill \langle b,C\rangle\in{\EuScript D}}{\Rightarrow}{\bigvee_{i<k}\ \varphi(a_i\,;b\,;C^{\delta+\varepsilon})}\medskip 

  \ceq{2.\hfill \langle b,C^{\delta+\varepsilon}\rangle\in{\EuScript D}}{\Leftarrow}{\bigvee_{i<k}\ \varphi(a_i\,;b\,;C)} 
\end{lemma}

\begin{proof}
  We define recursively the required parameters $a_i$ together with some auxiliary parameters $b_i$.
  The element $a_n$ is choosen such that

\ceq{3.\hfill \langle b,C^{\delta+\varepsilon}\rangle\in{\EuScript D}}{\Leftarrow}{\varphi(a_n\,;b\,;C)}\hfill for every $b\in{\EuScript U}^z$ and every $C$

and

\ceq{4.\hfill \langle b_i,C\rangle\in{\EuScript D}}{\Rightarrow}{\varphi(a_n\,;b_i\,;C^\delta)}\hfill for every $i<n$ and every $C$.\smallskip

This is possible because ${\EuScript D}$ is $\delta$-approximated from $\varepsilon$-below.
Note that (3) immediately guarantees (2).
Now, assume (1) fails for $k=n$, and choose $b_n$ and $C_n$ witnessing this.
Then, by Fact~\ref{fact_otto}, for some $\tilde C_n\cap C_n^{\delta+\varepsilon}=\varnothing$

\ceq{5.\hfill \langle b_n,C_n\rangle\in{\EuScript D}}{\&}{\bigwedge_{i<n}\ {\sim}\varphi(a_i\,;b_n\,;\tilde C_n)}%\smallskip

Suppose for a contradiction that the construction never ends.
Then, as (5) guarantees that $\langle b_i,C_i\rangle\in{\EuScript D}$ for every $i$, from (4) we otain $\varphi(a_n\,;b_i\,;C_n^\delta)$ for every $i<n$.
From (5) we also obtain ${\sim}\varphi(a_i\,;b_n\,;\tilde C_n)$, for every $i\le n$.
This contradicts $\varepsilon$-stability.
\end{proof}

\begin{lemma}
  Let $\varphi(x\,;z\,;X)$ be $\varepsilon$-stable.
  Assume that ${\EuScript D}$ is $\delta$-approximable by $\varphi(x\,;z\,;X)$.
  Let $m$ be maximal so that a sequence as in (3) of Definition/Theorem~\ref{defthm_epsilon_stable} exists.
  Let $\bar x=\langle x_i\ :\ i\le m\rangle$ where the $x_i$ are copies of $x$.
  Then the formula\smallskip

  \ceq{\hfill\sigma(\bar x\,;z\,;X)}{=}{\bigwedge_{i\le m}\ \varphi(x_i\,;z\,;X)}\smallskip

  $\delta$-approximate ${\EuScript D}$ from $\varepsilon$-below.
\end{lemma}

\begin{proof}
  Negate the claim and let $B$ witness that $\sigma(\bar x)$ does not $\delta$-approximate ${\EuScript D}$ from $\varepsilon$-below.
  Suppose that $a_0,\dots,a_{n-1}$ and $b_0,\dots,b_{n-1}$ have been defined.
  Choose $a_n$ such that for every $b\in B\cup\{b_0,\dots,b_{n-1}\}$ and every $C$

  \ceq{1.\hfill\langle b,C\rangle\in{\EuScript D}}{\Rightarrow}{\varphi(a_n\,;b\,;C^\delta)}\quad and

  \ceq{2.\hfill\langle b,C^{\varepsilon+\delta}\rangle\in{\EuScript D}}{\Leftarrow}{\varphi(a_n\,;b\,;C^\varepsilon)}
  
  Note that the latter implication is equivalent to: for every $C$ there is some  $\tilde C\cap C^\varepsilon=\varnothing$ such that 
  
  \ceq{3.\hfill \langle b,C^{\varepsilon+\delta}\rangle\notin{\EuScript D}}{\Rightarrow}{{\sim}\varphi(a_n\,;b\,;\tilde C)}.%\hfill  for every $b\in B\cup\{b_0,\dots,b_{n-1}\}$.

  Now, as the lemma is assumed to fail, we can choose $b_n$ and $C_n$ such that

  \ceq{4.\hfill \langle b_n,C_n^{\varepsilon+\delta}\rangle\notin{\EuScript D}}{\&}{\bigwedge_{i=0}^n\ \varphi(a_i\,;b_n\,;C_n)}

  Note that (4) ensure that $\langle b_i,C_i^{\delta+\varepsilon}\rangle\notin{\EuScript D}$ for every $i$.
  The there is some $\tilde C_i$ that witnesses (3) for $\langle b_i,C_i^{\delta+\varepsilon}\rangle\notin{\EuScript D}$.
  We claim that the procedure has to stop after $\le m$ steps. 
  In fact, from (3) we obtain ${\sim}\varphi(a_n\,;b_i\,;\tilde C_i)$ for every $i<n$.
  On the other hand, by (4) we have that $\varphi(a_i\,;b_n\,;C_n)$ for every $i<n$.
  Therefore, $\langle a_{m-i}\,;b_{m-i}\,;C_{m-i}\,;\tilde C_{m-i}\ :\ i\le m\rangle$ contradicts the maximality of $m$.
\end{proof}
