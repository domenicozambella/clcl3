\documentclass{amsproc}
\usepackage{subfiles}
\usepackage[utf8]{inputenc}
\usepackage[english]{babel}
\usepackage{comment}
\usepackage[alphabetic]{amsrefs}
\usepackage{tikz}
\usepackage{xcolor}
\usepackage{datetime2} 
\usepackage[colorlinks=true,linkcolor=blue,]{hyperref}
\usepackage{tcolorbox}
\usepackage{xstring}

\usepackage{stackengine}
\usepackage{scalerel}
\usepackage{mathtools}
\usepackage{calc}
\usepackage{amsthm}
\usepackage{thmtools}
\usepackage[framemethod=TikZ]{mdframed}
\usepackage{amssymb}
\usepackage{amsfonts}
\usepackage{euscript}
% \def\EuScript#1{\mathscr#1}
% \usepackage{fourier-otf}
\usepackage{fourier}
%\usepackage{palatino}
% \usepackage[sc]{mathpazo} % add possibly `sc` and `osf` options
%\usepackage{eulervm}
%\usepackage{bbm}
%\usepackage{latexsym}
% \usepackage{mathrsfs}
% \usepackage{stmaryrd}
% \usepackage{stix}
\def\dotminus{\stackon[.2ex]{$-$}{$.$}}
% \def\wneg{\stackon[-.2ex]{$\neg$}{$\neg$}}
\usepackage{dsfont}
% \newcommand*{\TakeFourierOrnament}[1]{{%
% \fontencoding{U}\fontfamily{futs}\selectfont\char#1}}
% \renewcommand*{\danger}{\TakeFourierOrnament{66}}
\parindent0ex
\parskip1.2ex
% \makeatletter
% \DeclareOldFontCommand{\rm}{\normalfont\rmfamily}{\mathrm}
% \DeclareOldFontCommand{\sf}{\normalfont\sffamily}{\mathsf}
% \DeclareOldFontCommand{\tt}{\normalfont\ttfamily}{\mathtt}
% \DeclareOldFontCommand{\bf}{\normalfont\bfseries}{\mathbf}
% \DeclareOldFontCommand{\it}{\normalfont\itshape}{\mathit}
% \DeclareOldFontCommand{\sl}{\normalfont\slshape}{\@nomath\sl}
% \DeclareOldFontCommand{\sc}{\normalfont\scshape}{\@nomath\sc}
% \makeatother

\newcommand{\mylabel}[1]{{#1}\hfill}
\renewenvironment{itemize}
  {\begin{list}{$\triangleright$}{%
  \setlength{\parskip}{0mm}
  \setlength{\topsep}{.1\baselineskip}
  \setlength{\rightmargin}{0mm}
  \setlength{\listparindent}{0mm}
  \setlength{\itemindent}{0mm}
  \setlength{\labelwidth}{3ex}
  \setlength{\itemsep}{.1\baselineskip}
  \setlength{\parsep}{.1\baselineskip}
  \setlength{\partopsep}{0mm}
  \setlength{\labelsep}{1ex}
  \setlength{\leftmargin}{\labelwidth+\labelsep}
  \let\makelabel\mylabel}}{%
\end{list}}


\newtheoremstyle{mio}% name
     {2\parskip}     % Space above
     {2\parskip}     % Space below
     {}%         Body font
     {}%         Indent amount (empty = no indent, \parindent = para indent)
     {\bfseries}% Thm head font
     {}%        Punctuation after thm head
     {1ex}%     Space after thm head: " " = normal interword space;
           %   \newline = linebreak
     {\llap{\thmnumber{#2}\hskip0.9ex}\thmname{#1}\thmnote{\bfseries{}#3}}% Thm head spec (can be left empty, meaning `normal')

\newcounter{thm}

\tcbset{
  mythm/.style={
    colback=black!5!white,
    colframe=white!50!black,
    extrude right by=2ex,
    extrude left by=6.5ex,
    % before=\par\noindent,
    % after=\par\noindent,
    top=0.7ex,
    bottom=1.5ex,
    right=2ex,
    left=6.5ex,
    boxsep=0pt,
    % parbox=false,
    before skip=\baselineskip,
    after skip=\baselineskip,
  },
}
\theoremstyle{mio}
\newtheorem{theorem}[thm]{Theorem}\tcolorboxenvironment{theorem}{mythm}
\newtheorem{corollary}[thm]{Corollary}\tcolorboxenvironment{corollary}{mythm}
\newtheorem{proposition}[thm]{Proposition}\tcolorboxenvironment{proposition}{mythm}
\newtheorem{lemma}[thm]{Lemma}\tcolorboxenvironment{lemma}{mythm}
\newtheorem{fact}[thm]{Fact}\tcolorboxenvironment{fact}{mythm}
\newtheorem{definition}[thm]{Definition}\tcolorboxenvironment{definition}{mythm}
\newtheorem{assumption}[thm]{Assumption}\tcolorboxenvironment{assumption}{mythm}
\newtheorem{void}[thm]{}\tcolorboxenvironment{void}{mythm}
\newtheorem{notation}[thm]{Notation}\tcolorboxenvironment{notation}{mythm}
\newtheorem{note}[thm]{Note}\tcolorboxenvironment{note}{mythm}
\newtheorem{remark}[thm]{Remark}\tcolorboxenvironment{remark}{mythm}
\newtheorem{definition_theorem}[thm]{Definition\,/\,Theorem}\tcolorboxenvironment{definition_theorem}{mythm}
\newtheorem{exercise}[thm]{Exercise}
\newtheorem{example}[thm]{Example}
\newtheorem{question}[thm]{Question}


\makeatletter
\providecommand{\proofNameStyle}{\bfseries}
\renewenvironment{proof}[1][\proofname]{\par
  \pushQED{\qed}%
  \normalfont% \topsep6\p@\@plus6\p@\relax
  % \vspace*{-\baselineskip}
  \trivlist
  \item[\hskip\labelsep
        \proofNameStyle
    #1\@addpunct{.}]\ignorespaces
}{%
  \popQED\endtrivlist\@endpefalse
}
\makeatother



\renewcommand*{\emph}[1]{%
   \smash{\tikz[baseline]\node[rectangle, fill=teal!25, rounded corners, inner xsep=0.5ex, inner ysep=0.2ex, anchor=base, minimum height = 2.7ex]{\strut #1};}}

%%%%%%% GETCOMMIT
% \newcommand\dotGitHEAD{}
% \newcommand\branch{}


% \makeatletter\let\myfilehandle\@inputcheck\makeatother

% \openin\myfilehandle=.git/HEAD\relax

% \begingroup\endlinechar-1
%   \global\read\myfilehandle to \dotGitHEAD
% \endgroup
% \closein\myfilehandle

% \newcommand\GetBranch{}
% \def\GetBranch ref: refs/heads/#1\relax{\renewcommand{\branch}{#1}}

% \expandafter\GetBranch\dotGitHEAD\relax

% \openin\myfilehandle=.git/refs/heads/\branch\relax

% \begingroup\endlinechar-1
%   \global\read\myfilehandle to \commit
% \endgroup
% \closein\myfilehandle

% \makeatother

\linespread{1.1}
\author{Silvia Barbina}
\author{Riccardo Camerlo}
\author{Domenico Zambella}


\address[Silvia Barbina]{Sezione di Matematica, Universit\`{a} di Camerino}
\email[Silvia Barbina]{silvia.barbina@unicam.it}
\address[Riccardo Camerlo]{Dipartimento di Matematica, Universit\`{a} di Genova}\email[Riccardo Camerlo]{riccardo.camerlo@unige.it}
\address[Domenico Zambella]{Dipartimento di Matematica, Universit\`{a} di Torino}
\email[Domenico Zambella]{domenico.zambella@unito.it}

\thanks{Silvia Barbina was partially supported by PRIN2022 \textit{Models, sets and classifications}, prot.\@ 2022TECZJA.
Riccardo Camerlo was partially supported by the MUR excellence department project awarded to the Department of Mathematics of the University of Genoa, CUP D33C23001110001. 
Domenico Zambella was partially supported by PRIN2022 \textit{Logical methods in combinatorics}, prot.\@ 2022BXH4R5}

\subjclass[2020]{03C66, 03C45, 03C68}

\begin{document}
\hfill 25 October 2025\bigskip

This paper concerns notions of local stability for \textit{structures with a standard sort.}
These are a classical framework for continuous structures recently introduced in \cite{clcl} as an alternative to~\cite{BBHU}.

Continuous model theory~\cite{BBHU} has replaced Henson-Iovino logic~\cite{HI} as a formalism to study the model theory of continuous structures. 
One of the first articles on the subject, \cite{BU}, defines local stability within continuous logic.
As noted by Henson, this definition escapes~\cite{HI}'s approach.

We examine three variants of the order property in the context of \cite{clcl}, and for each we prove a suitable version of the classical theorem that says that externally definable sets are definable.

If it is appropriate to indicate a specific editor, our preference would be Krzysztof Krupi\'nski.

Thank you for your consideration.
%%%%%%%%%%%%%%%%%%%%%
%%%%%%%%%%%%%%%%%%%%%
%%%%%%%%%%%%%%%%%%%%%
%%%%%%%%%%%%%%%%%%%%%
%%%%%%%%%%%%%%%%%%%%%
%%%%%%%%%%%%%%%%%%%%%
%%%%%%%%%%%%%%%%%%%%%

\BibSpec{arXiv}{%
  +{}{\PrintAuthors}{author}
  +{,}{ \textit}{title}
  +{}{ \parenthesize}{date}
  +{,}{ arXiv:}{eprint}
  +{,}{ } {note}
  % +{,}{ \url}
}

\BibSpec{webpage}{%
  +{}{\PrintAuthors} {author}
  +{,}{ \textit} {title}
  +{,}{ } {portal}
  +{}{ \parenthesize} {date}
  +{,}{ } {doi}
  +{,}{ } {note}
  +{.}{ } {transition}
}
\begin{bibdiv}
\begin{biblist}[]\normalsize

%   \bib{A}{article}{
%     label={A},
%     author={Auslander, Josef},
%     title={Topological Dynamics},
%     journal={Scholarpedia},
%     date={2008},
%     note={\href{http://scholarpedia.org/article/Topological_dynamics}{scholarpedia.3449}},
% }\smallskip
\bib{clcl}{article}{
    label={ABBMZ},
    author = {Agostini, Claudio},
    author = {Baratella, Stefano},
    author = {Barbina, Silvia},
    author = {Motto Ros, Luca},
    author = {Zambella, Domenico},
    journal={Bull. Iran. Math. Soc.},
    volume={51},
    date = {2025},
    title = {Continuous logic in a classical setting},
    note={\href{https://arxiv.org/abs/2402.01245}{arXiv:2402.01245}},
  }\smallskip
\bib{BBHU}{article}{
  label={BBHU},
  author={Ben Yaacov, Ita\"{\i}},
  author={Berenstein, Alexander},
  author={Henson, C. Ward},
  author={Usvyatsov, Alexander},
  title={Model theory for metric structures},
  conference={
      title={Model theory with applications to algebra and analysis. Vol. 2},
  },
  book={
      series={London Math. Soc. Lecture Note Ser.},
      volume={350},
      publisher={Cambridge Univ. Press, Cambridge},
  },
  %  isbn={978-0-521-70908-8},
  date={2008},
  pages={315--427},
  %  review={\MR{2436146}},
  %  doi={10.1017/CBO9780511735219.011},
}\smallskip
\bib{BU}{article}{
   label={BU},
   author={Ben Yaacov, Ita\"i},
   author={Usvyatsov, Alexander},
   title={Continuous first order logic and local stability},
   journal={Trans. Amer. Math. Soc.},
   volume={362},
   date={2010},
  %  number={10},
   pages={5213--5259},
  %  issn={0002-9947},
  %  review={\MR{2657678}},
  %  doi={10.1090/S0002-9947-10-04837-3},
  note={\href{https://arxiv.org/abs/0801.4303}{arXiv:0801.4303}},
}

\bib{HI}{article}{
  label={HI},
  author={Henson, C. Ward},
  author={Iovino, Jos\'{e}},
  title={Ultraproducts in analysis},
  conference={
    title={Analysis and logic},
    address={Mons},
    date={1997},
   },
   book={
      series={London Math. Soc. Lecture Note Ser.},
      volume={262},
      publisher={Cambridge Univ. Press, Cambridge},
   },
   date={2002},
   pages={1--110},
  %  review={\MR{1967834}},
}
% \bib{Z}{article}{
%   label={Z},
%   author = {Zambella, Domenico},
%   title = {Standard analysis},
%   % doi = {10.48550/arXiv.2311.15711},
%   note={\href{https://arxiv.org/abs/2311.15711}{arXiv:2311.15711}},
%   date = {2023},
% }
%%%%%%%%%%%%%%%%%%%%%%%%%%%%%%%%%%%
%%%%%%%%%%%%%%%%%%%%%%%%%%%%%%%%%%%
%%%%%%%%%%%%%%%%%%%%%%%%%%%%%%%%%%%
%%%%%%%%%%%%%%%%%%%%%%%%%%%%%%%%%%%
%%%%%%%%%%%%%%%%%%%%%%%%%%%%%%%%%%%
%%%%%%%%%%%%%%%%%%%%%%%%%%%%%%%%%%%
% \section{Continuous group actions}
\end{biblist}
\end{bibdiv}
\end{document}