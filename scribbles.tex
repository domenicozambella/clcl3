% !TEX root = clcl3.tex
\section{Scribbles}


%%%%%%%%%%%%%%%%%%%%%%%%%
%%%%%%%%%%%%%%%%%%%%%%%%%%
%%%%%%%%%%%%%%%%%%%%%%%%%%
%%%%%%%%%%%%%%%%%%%%%%%%%%
%%%%%%%%%%%%%%%%%%%%%%%%%%
\section{Stable functions}
\def\medrel#1{\parbox{5ex}{\hfil $#1$}}
\def\ceq#1#2#3{\parbox[t]{33ex}{$\displaystyle #1$}\medrel{#2}{$\displaystyle #3$}}


\begin{fact}
  The following are equivalent
  \begin{itemize}
    \item [1] ${\EuScript D\!}_f$ is approximable by $\tau(x\,;z)\in X$
    \item [2] for every finite $B\subseteq{\EuScript U}^{z}$ there is an $a\in{\EuScript U}^{x}$ such that\smallskip
    
    \noindent\kern-\labelwidth\kern-\labelsep
    \ceq{\hfill\langle f(b),\ \tau(a\,;b)\rangle}{\in}{\varepsilon}\hfill for every $b\in B$ and every $\varepsilon$.
  \end{itemize}
  
\end{fact}

In this section we require that ${\EuScript D}$ has the following property

% \ceq{}{}{}
For every $b\in{\EuScript U}^z$ and every $C\in K(S)$, there is some $\tilde C\cap C=\varnothing$ such that

be the set of all $z$-tuples of elements of ${\EuScript U}$.
We say that $f:{\EuScript U}^{z}\to S$ is \emph{$\varepsilon$-approximable\/} by $\tau(x\,;z)$ if for every finite $B\subseteq{\EuScript U}^z$ and there is an $a\in{\EuScript U}^x$ such that

\ceq{\hfill \big|f(b)\;,\tau(a\,;b)\big|}{\le}{\varepsilon}\hfill for every $b\in B$.

We say \emph{approximable\/} for $\varepsilon$-approximable for every $\varepsilon>0$.

\begin{theorem}
  Let $S=[0,1]$.
  Let $\tau(x\,;z)$ be a stable type-definable function.
  Let $f:{\EuScript U}^{z}\to S$ be approximable by $\tau(x\,;z)$.
  Then $f$ is type-definable. 
\end{theorem}

\begin{proof}
  The following three lemmas show that if $\tau(x\,;z)$ is $\varepsilon$-approximable, then there are some $\langle a_{i,j}\ :\ i<n,\ j<\omega\rangle$ such that\medskip

  \ceq{\hfill\big|f(b)\ -\ \max_{i=0,\dots,n-1}\ \inf_{j<\omega}\ \tau(a_{i,j}\,;b)\big|}{\le}{3\varepsilon}\hfill for every $b\in{\EuScript U}^z$.

  Then $f$ is the limit of type-definable functions, hence type-definable.
\end{proof}

We say that $f:{\EuScript U}^{x;z}\to[0,1]$ is $\varepsilon$-approximable \emph{from below\/} if the parameter $a\in{\EuScript U}^x$ that witnesses approximability satisfies $\tau(a\,;b)\le f(b)+\varepsilon$ for every $b\in {\EuScript U}^z$.

\begin{lemma}
  Assume that the $f$ in the theorem is $\varepsilon$-approximable from below.
  Then there are some $\langle a_i\:\ i< n\rangle$ such that\smallskip

  \ceq{\hfill\big|f(b)\ -\ \max_{i=0,\dots,n-1}\ \tau(a_i\,;b)\big|}{\le}{\varepsilon}\hfill for every $b\in{\EuScript U}^z$.
\end{lemma}

\begin{proof}
  Negate the lemma.
  We construct inductively a sequence $\langle a_i\,;b_i\ :\ i<\omega\rangle$ that contradicts the stability of $\tau(x\,;z)$.
  Then, let $b_n$ be such that (if $n=0$ pick $b_n$ arbitrarily)

  \ceq{1.\hfill\big|f(b_n)\ -\max_{i=0,\dots,n-1}\ \tau(a_i\,;b_n)\big|}{>}{\varepsilon}

  By approximability from below, for some $a_n$ we have

  \ceq{2.\hfill\tau(a_n\,;b_i)\ -\ f(b_i)}{\le}{\varepsilon}\hfill for every $i\le n$

  \ceq{3.\hfill\tau(a_n\,;b)}{\le}{ f(b)+\varepsilon}\hfill for every $b\in{\EuScript U}^{z}$.

  We prove instability by showing that for every $i<n$

  \ceq{4.\hfill \tau(a_n\,;b_i)\ -\ \tau(a_i\,;b_n)}{>}{\varepsilon}

  Note that by virtue of (3) 

  \ceq{\hfill \max_{i=0,\dots,n-1}\ \tau(a_i\,;b_n)-f(b_n)}{\le}{\varepsilon}

  Then (1) ensures that

  \ceq{1'.\hfill f(b_n)\ -\max_{i=0,\dots,n-1}\ \tau(a_i\,;b_n)}{>}{\varepsilon}

  Using ($1'$) we obtain

  \ceq{\hfill \tau(a_n\,;b_i)-\tau(a_i\,;b_n)}{>}{\tau(a_n\,;b_i)-f(b_n)+\varepsilon}

  then from (2) we obtain

  \ceq{}{>}{f(b_i)-f(b_n)+2\varepsilon}

  By refining the sequence we can assume that $|f(b_i)-f(b_n)|<\varepsilon$ for every $i,n\in\omega$.
  This yields (4).
\end{proof}

\begin{lemma}
  Under the assumptions of the theorem. 
  Let $\bar x=\langle x_i\ :\ i<\omega\rangle$ where the $x_i$ are copies of $x$.
  Then the function\smallskip

  \ceq{\hfill\sigma(\bar x\,;z)}{=}{\inf_{i<\omega}\ \tau(x_i\,;z)}

  $3\varepsilon$-approximates $f$ from below.
\end{lemma}

\begin{proof}
  Negate the claim and let $B$ witness that $\sigma(\bar x\,;z)$ does not $\varepsilon$-approximate $f$ from below.
  We construct inductively a sequence $\langle a_i\,;b_i\ :\ i<\omega\rangle$ that contradicts the stability of $\tau(x\,;z)$.
  Then, let $a_n$ be such that

  \ceq{1.\hfill\big|\tau(a_n\,;b)\ -\ f(b)\big|}{\le}{\varepsilon}\hfill for every $b\in B\cup\{b_0,\dots,b_n\}$

  then let $b_n$ be such that

  \ceq{2.\hfill\min_{i=0,\dots,n}\tau(a_i\,;b_n)}{>}{ f(b_n)+3\varepsilon.}

  We show that the sequence we constructed contradicts stability.
  In fact, by (2)

  \ceq{\hfill\tau(a_i\,;b_n)\ -\ \tau(a_n\,;b_i)}{>}{f(b_n)+3\varepsilon-\tau(a_n\,;b_i).}

  Then, by (1)

  \ceq{\hfill\tau(a_i\,;b_n)\ -\ \tau(a_n\,;b_i)}{>}{f(b_n)+3\varepsilon-f(b_i) - \varepsilon}

  As in the previous lemma, after refining the sequence we we obtain a contradiction.
\end{proof}

\begin{lemma}
  If $\tau(\bar x\,;z)$ is stable, then $\sigma(\bar x\,;z)$ in the previous lemma is stable. 
\end{lemma}

\begin{proof}
  It suffices to show that

  \ceq{\hfill\sigma(x_1,x_2\,;z)}{=}{\min_{i=1,2}\ \tau(x_i\,;z)}

  is stable.
  Suppose not.
  Pick $\langle a_{1,i},a_{2,i}\,;b_i\ :\ i<\omega\rangle$ and $\varepsilon>0$  such that for every $i<j<\omega$

  \ceq{\hfill\big|\sigma(a_{1,i},a_{2,i}\,;b_j)-\sigma(a_{1,j},a_{2,j}\,;b_i)\big|}{\ge}{\varepsilon}

  By Ramsey's theorem, we can refine the sequence so that one of the following holds for every $i<j$

\ceq{\hfill\sigma(a_{1,i},a_{2,i}\,;b_j)-\sigma(a_{1,j},a_{2,j}\,;b_i)}{\ge}{\varepsilon}
  
\ceq{\hfill-\sigma(a_{1,i},a_{2,i}\,;b_j)+\sigma(a_{1,j},a_{2,j}\,;b_i)}{\ge}{\varepsilon}

Assume the first~--~the second case is symmetric.
Again by Ramsey's theorem, we can refine the sequence so that for every $i<j$ either  $\tau(a_{1,i}\,;b_j)\le\tau(a_{2,i}\,;b_j)$ or conversely $\tau(a_{1,i}\,;b_j)\ge\tau(a_{2,i}\,;b_j)$.
In the first case we obtain

\ceq{\hfill\tau(a_{1,i}\,;b_j)-\tau(a_{2,j}\,;b_i)}{\ge}{\varepsilon}

in the second case we obtain

\ceq{\hfill\tau(a_{2,i}\,;b_j)-\tau(a_{1,j}\,;b_i)}{\ge}{\varepsilon.}

In both cases we contradict the stability of $\tau(x\,;z)$.
\end{proof}



% If $\varphi(X)\in{\EuScript F}_{\sf X}$ and ${\EuScript D}=\{C:\varphi(C)\}$ then ${\sim}{\EuScript D}=\{C:{\sim}\varphi(C)\}$.
\begin{comment}
%%%%%%%%%%%%%%%%%%%%%%%%%%%%%%%
%%%%%%%%%%%%%%%%%%%%%%%%%%%%%%%
%%%%%%%%%%%%%%%%%%%%%%%%%%%%%%%
%%%%%%%%%%%%%%%%%%%%%%%%%%%%%%%
\section{Attempt 1}
\def\medrel#1{\parbox{5ex}{\hfil $#1$}}
\def\ceq#1#2#3{\parbox[t]{22ex}{$\displaystyle #1$}\medrel{#2}{$\displaystyle #3$}}

Let $\varphi(x\,;z\,;X)$ be \emph{type-definable\/} that is, definable by an infinite (small) conjunction of formulas in ${\EuScript F}_{\sf X}({\EuScript U})$.
The variables of sort ${\sf H}$ are partitioned in two tuples $x\,;z$ which may be infinite.

We say that $\varphi(x\,;z\,;X)$ is \emph{unstable\/} if there is a sequence $\langle a_i\,;b_i\ :\ i<\omega\rangle$ such that for some $C\cap\tilde C=\varnothing$ and every $i<n<\omega$

\ceq{\hfill \varphi(a_n\,;b_i\,;C)}{\wedge}{{\sim}\varphi(a_i\,;b_n\,;\tilde C)}

We say that $\varphi(x\,;z\,;X)$ is \emph{stable\/} if it is not unstable.
\setcounter{thm}{19}
\begin{theorem}[ (conjecture)]
  The following are equivalent
  \begin{itemize}
    \item [1.] $\varphi(x\,;z\,;X)$ is stable
    \item [2.] there is no sequence $\langle a_i\,;b_i\,;C_i\ :\ i<\omega\rangle$ such that for some $C$, some neighboorhood $C'$ of $C$,  and every $i<n<\omega$

\ceq{\hfill \varphi(a_n\,;b_i\,;C)}{\wedge}{{\sim}\varphi(a_i\,;b_n\,;C_n)}

and $C'\cap C_n=\varnothing$.
\end{itemize}
\end{theorem}


If ${\EuScript D}\subseteq{\EuScript U}^z\times K(S)$ then ${\sim}{\EuScript D}$ is obtained by applying the definition in Section~\ref{pseudocomplement} to all fibers of ${\EuScript D}$.



\begin{theorem}[ (conjecture)]
  Let $\varphi(x\,;z\,;X)$ be stable.
  Let ${\EuScript D}\subseteq {\EuScript U}^z\times K(S)^{|X|}$ be involutive and approximable by $\varphi(x\,;z\,;X)$.
  Then there are some $\langle a_{i,j}\ :\ i\le n,\ j<\omega\rangle$ such that for every $b\in{\EuScript U}^z$\medskip

  \ceq{\hfill\langle b,C\rangle\in{\EuScript D}}{\Leftrightarrow}{\bigvee_{i\le n}\ \bigwedge_{j<\omega}\ \varphi(a_{i,j}\,;b\,;C)}
\end{theorem}

\begin{proof}
  
\end{proof}

We say that ${\EuScript D}$ is approximable by $\varphi(x\,;z\,;X)$ if \emph{from below\/} if in the definition above the implication in (2) holds for every $b\in{\EuScript U}^z$ and $C\in K(S)$.

\begin{lemma}
  Under the assumptions of the theorem.
  If ${\EuScript D}$ is approximable from below then there are some $\langle a_i\ :\ i\le n\rangle$ such that for every $b\in{\EuScript U}^z$\medskip

  \ceq{\hfill\bigvee_{i\le n}\ \varphi(a_i\,;b\,;C)}{\Leftrightarrow}{\langle b, C\rangle\in{\EuScript D}} 
\end{lemma}

\begin{proof}
The sequences $a_0,\dots,a_n$ is defined recursively together with an auxiliary sequence $b_0,\dots,b_{n-1}$.
The elements in the sequence $a_0,\dots,a_n$ are choosen to obtain 

\ceq{1.\hfill\bigvee_{i=0}^n\ \varphi(a_i\,;b\,;C)}{\Rightarrow}{\langle b, C\rangle\in{\EuScript D}}\hfill for every $b\in{\EuScript U}^z$ and every $C$

and

\ceq{2.\hfill\langle b_i, C\rangle\in{\EuScript D}}{\Rightarrow}{\varphi(a_n\,;b_i\,;C)}\hfill for every $i<n$ and every $C$.\smallskip

This is possible because ${\EuScript D}$ is approximated from below.
Now choose $b_n$ and $C_n$ such that

\ceq{3.\hfill\bigwedge_{i=0}^n\ {\sim}\varphi(a_i\,;b_n\,;C_n)\ }{\text{and}}{\ \langle b_n,C_n\rangle\notin{\sim}{\EuScript D}}%\smallskip

If such $b_n$ and $C_n$ do not exist, then 

\ceq{\hfill{\sim}\bigvee_{i=0}^n\ \varphi(a_i\,;b\,;C)\ }{\Rightarrow}{\ \langle b,C\rangle\in{\sim}{\EuScript D}}\hfill for every $b\in{\EuScript U}^z$ and every $C$%\smallskip

which is equivalent to the converse of (1) and proves the lemma.
Suppose for a contradiction that the construction never ends.
Then, from (2) we otain $\varphi(a_n\,;b_i\,;C)$ for every $i<n$ and from (3) we obtain ${\sim}\varphi(a_i\,;b_n\,;C_n)$, again for every $i<n$.
This contradicts stability.
\end{proof}

\begin{lemma}
  Under the assumptions of the theorem.
  Let $\bar x=\langle x_m\ :\ i<\omega\rangle$ where the $x_i$ are copies of $x$ and $m$ is the maximal .
  Then the formula\medskip

  \ceq{\hfill\sigma(\bar x\,;z\,;X)}{=}{\bigwedge_{i<m}\ \varphi(x_i\,;z\,;X)}

  approximate ${\EuScript D}$ from below.
\end{lemma}

\begin{proof}
  Negate the claim and let $B$ and $C$ witness that $\sigma(\bar x)$ does not approximate ${\EuScript D}$ from below.
  Suppose that $a_0,\dots,a_{n-1}$ and $b_0,\dots,b_{n-1}$ have been defined.
  Choose $a_n$ such that 

  \ceq{1.\hfill \langle b,C\rangle\in {\EuScript D}}{\Rightarrow}{\varphi(a_n\,;b\,;C)}

  \ceq{2.\hfill \langle b,C\rangle\in{\sim}{\EuScript D}}{\Rightarrow}{{\sim}\varphi(a_n\,;b\,;C)}\hfill for every $b\in B\cup\{b_0,\dots,b_{n-1}\}$.

  Now, as the lemma is assume to fail, we can choose $b_n$ such that

  \ceq{3.\hfill\bigwedge_{i=0}^n\ \varphi(a_i\,;b_n\,;C)\ }{\text{and}}{\ \langle b_n,C\rangle\notin{\EuScript D}.}

  Moreover, as ${\EuScript D}$ is involutive, there is $C_n\cap C=\varnothing$ such that $\langle b_n,C_n\rangle\in{\sim}{\EuScript D}$.
  
  We claim that the sequence we obtain contradicts stability.
  In fact, the construction guarantees that $\langle b_i,C_i\rangle\in{\sim}{\EuScript D}$ for every $i$, then from (2) we obtain that ${\sim}\varphi(a_n\,;b_i\,;C_i)$ for every $i<n$.
  On the other hand, by (1) we have that $\varphi(a_n\,;b_i\,;C)$ for every $i<n$.
\end{proof}

\begin{lemma}
  If $\varphi(x\,;z\,;C)$ is stable then $\sigma(\bar x\,;z\,;X)$ in the previous lemma is stable.
\end{lemma}

\begin{proof}
  Assume $\sigma(x_1,x_2\,;z\,;X)$ in unstable.
  Let $\langle a_{1,i},a_{2,i}\,;b_i\,;C_i\ :\ i<\omega\rangle$ be a sequence witnessing instability.
  then for every $i<n<\omega$ either ${\sim}\varphi(a_{1,n}\,;b_i\,;C_i)$ or ${\sim}\varphi(a_{2,n}\,;b_i\,;C_i)$

\end{proof}

%%%%%%%%%%%%%%%%%%%%%%%%%%%%%%%
%%%%%%%%%%%%%%%%%%%%%%%%%%%%%%%
%%%%%%%%%%%%%%%%%%%%%%%%%%%%%%%
%%%%%%%%%%%%%%%%%%%%%%%%%%%%%%%
\section{Externally definable sets}

\emph{global $\varphi(x\,;z\,;X)$-type\/} is a maximally consistent set of formulas that are instances of $\varphi(x\,;b\,;X)$ and/or ${\sim}\varphi(x\,;b\,;X)$ for some $b\in{\EuScript U}^{|z|}$.

% \begin{fact}
%   Let $p(x)$ be a global $\varphi(x\,;z\,;X)$-type.
% \end{fact}

Let $p(x)$ be a global $\varphi(x\,;z\,;X)$-type.
We define 

\ceq{\hfill\emph{${\EuScript D}_{p,\varphi,C}$}}{=}{\big\{b\in{\EuScript U}^{z}\ :\ \varphi(x\,;b\,;C)\in p\big\}}

\begin{fact}\ \vskip-0.5\baselineskip
  
  \ceq{\hfill\neg{\EuScript D}_{p,\varphi,C}}{=}{\bigcup\Big\{{\EuScript D}_{p,{\sim}\varphi,\tilde C}\ \ :\ \ \tilde C\cap C=\varnothing\Big\}}
\end{fact}

\begin{proof}
  ($\subseteq$).
  Let $b\notin{\EuScript D}_{p,\varphi,C}$.
  Let $A\subseteq{\EuScript U}$ be such that every instance of $\varphi(x\,;b\,;X)$ and ${\sim}\varphi(x\,;b\,;X)$ that is consistent with $p(x)\restriction A$ is finitely consistent with $p(x)$.
  Finally pick $a\models p(x)\restriction A$.
  As $\neg\varphi(a,b,C)$, from Fact~\ref{fact_maximal} we obtain that  ${\sim}\varphi(a\,;b\,;\tilde C)$ holds for some $\tilde C\cap C=\varnothing$.
  Then ${\sim}\varphi(x\,;b\,;\tilde C)$ is consistent with $p(x)\restriction A$.
  By maximality of $p(x)$, we have ${\sim}\varphi(x\,;b\,;\tilde C)\in p$.
  Then $b\in{\EuScript D}_{p,{\sim}\varphi,\tilde C}$.
  
  ($\supseteq$).
  Similar.
\end{proof}

\begin{fact}
  Let $\varphi(x\,;z\,;X)$ and $C$ be given.
  Then for every ${\EuScript D}\subseteq {\EuScript U}^{z}$ the following are equivalent
  \begin{itemize}
    \item [1.] ${\EuScript D}={\EuScript D}_{p,\varphi,C}$ for some global $\varphi(x\,;z\,;X)$-type $p(x)$
    \item [2.] for every finite set $B\subseteq{\EuScript U}^{z}$ there is an $a\in{\EuScript U}^{x}$ such that
    
    \ceq{\hfill B\ \cap\ {\EuScript D}}{=}{B\ \cap\ \varphi(a\,;{\EuScript U}^x\,;C).}

  \end{itemize}
\end{fact}

\begin{proof}
  Let $B_0=B\smallsetminus{\EuScript D}$ and $B_1=B\cap{\EuScript D}$.

  1$\Rightarrow$2. 
  By the above fact, there is a $\tilde C\cap C=\varnothing$ such that $b\in{\EuScript D}_{p,\sim\varphi,\tilde C}$  form every $b\in B_0$.
  Let $a$ realize the type containing $\varphi(x\,;b\,;C)$ for every $b\in B_1$ and ${\sim}\varphi(x\,;b\,;\tilde C)$ for every $b\in B_0$. 

  2$\Rightarrow$1. 
  There is a set $\tilde C\cap C=\varnothing$ such that ${\sim}\varphi(x\,;b\,;\tilde C)$  for every $b\in B_0$.
  Let $p(x)$ be any global $\varphi(x\,;z\,;X)$-type containing $\varphi(x\,;b\,;C)$ for every $b\in B_1$ and ${\sim}\varphi(x\,;b\,;\tilde C)$ for every $b\in B_0$.
  Such type exists as (2) guarantees consistency.
\end{proof}

Let ${\EuScript D}\subseteq {\EuScript U}^{z}\times S^{\scriptscriptstyle|X|}$.
For $\alpha\in S^{\scriptscriptstyle|X|}$, we write ${\EuScript D}_\alpha$ for the set of all $b\in{\EuScript U}^{z}$ such that $\langle b,\alpha\rangle\in{\EuScript D}$.
We say that ${\EuScript D}$ is \emph{externally definable\/} by $\varphi(x\,;z\,;X)$ if there is a global $\varphi(x\,;z\,;X)$-type $p(x)$ such that for every $\alpha$

\ceq{\hfill{\EuScript D}_\alpha}{=}{\bigcap\big\{{\EuScript D}_{p,\varphi,C}\ :\ \alpha\in C^\circ\big\}}

We say that ${\EuScript D}$ is \emph{approximable\/} by $\varphi(x\,;z\,;X)$ if for every $\alpha\in S^{\scriptscriptstyle|X|}$ and every finite $B\subseteq{\EuScript U}^{z}$ there is an $a\in{\EuScript U}^{|x|}$ such that

\ceq{\hfill B\ \cap\ {\EuScript D}_\alpha}{=}{B\ \cap\ \bigcap\big\{\varphi(a\,;{\EuScript U}^{z}\,;C)\ :\ \alpha\in C^\circ\big\}}

\begin{fact}
  Let $\varphi(x\,;z\,;X)$ be given.
  Then, for every ${\EuScript D}\subseteq {\EuScript U}^{z}\times S^{\scriptscriptstyle|X|}$ the following are equivalent
  \begin{itemize}
    \item [1.] ${\EuScript D}$ is externally definable by $\varphi(x\,;z\,;X)$
    \item [2.] ${\EuScript D}$ is approximated by $\varphi(x\,;z\,;X)$.
  \end{itemize}
\end{fact}

We say that ${\EuScript D}$ is approximable \emph{from below\/} if for every $B\subseteq{\EuScript D}$ there is
an $a\in{\EuScript U}^{|x|}$ such that

\ceq{\hfill B}{\subseteq}{\varphi(a\,;{\EuScript U}^z\,;C)}\medrel{\subseteq}${\EuScript D}$.



\vskip3ex
Garbage after this line
\vskip-\baselineskip
\hrulefill
\vskip3ex

We define ${\EuScript D}_p\subseteq {\EuScript U}^{|z|}\times S^{\scriptscriptstyle|X|}$ as follows

\ceq{\hfill{\EuScript D}_{p}}{=}{\Big\{\,\langle b,\alpha\rangle\ :\  \varphi(x\,;b\,;C)\in p\ \textrm{ for every }C\textrm{ neighborhood of }\alpha\Big\}.}

% \ceq{\hfill\tilde {\EuScript D}_{p}}{=}{\Big\{\langle b,\alpha\rangle\ :\  {\sim}\varphi(x\,;b\,;\tilde C)\in p \textrm{ for every }C\rotatebox[origin=c]{180}{$\notin$}\alpha\Big\}.}

\begin{fact}\ \vskip-0.5\baselineskip

  \ceq{\hfill\neg {\EuScript D}_{p}}{=}{\Big\{\,\langle b,\alpha\rangle\ :\  {\sim}\varphi(x\,;b\,;\tilde C)\in p\ \textrm{  for every }\tilde C\,\rotatebox[origin=c]{180}{$\notin$}\,\alpha\;\Big\}.}
\end{fact}

\begin{proof}
  Let $a\models p(x)\restriction b$.
  Let $C$ be the intersection of all neighborhoods of $\alpha$ such that $\varphi(a\,;b\,;C')$.
  Assume $\langle b,\alpha\rangle\notin{\EuScript D}_p$, in other words $\varphi(a\,;b\,;C)$, then $C$ is a neighborhood of $\alpha$.
  By Fact~\ref{fact_maximal} we have $\varphi(a\,;b\,;C)$.
  Let $\tilde C\subseteq C\smallsetminus\{\alpha\}$ be such that $C$ is a neighborhood of $\tilde C$.
  Then $\neg\varphi(a\,;b\,;\tilde C)$.
  Therefor by the definition of ${\EuScript D}_p$.

  Then we can find $\tilde C$ such that $\alpha\notin\tilde C$ and $C$ is a neighborhood of $\tilde C$.
  contains two dijoint compacts
  Then ${\sim}\varphi(x\,;b\,;\tilde C')\in p$ by the definition of ${\EuScript D}_p$.


  Then ${\sim}\varphi(x\,;b\,;\tilde C')\in p$ for every neighborhood $\tilde C'$ of $\alpha$ disjoint from $C'$.
  Then for every neighborhood $C$ of $\alpha$, we have $\varphi(x\,;b\,;C)\in p$.
\end{proof}

We say that ${\EuScript D}$ is \emph{externally defined\/} by $p(x)$ and $\varphi(x\,;z\,;X)$ if ${\EuScript D}_p\subseteq{\EuScript D}$ and $\tilde{\EuScript D}_p\subseteq\neg{\EuScript D}$.

We say that ${\EuScript D}\subseteq {\EuScript U}^{|z|}\times S^{\scriptscriptstyle|X|}$ is \emph{approximated\/} by $\varphi(x\,;z\,;X)$ if for every finite $B\subseteq {\EuScript U}^{|z|}\times S^{\scriptscriptstyle|X|}$ there is an $a\in{\EuScript U}^{|x|}$ such that 

\ceq{\hfill \langle b,\alpha\rangle\in B\cap{\EuScript D}}{\Rightarrow}{\varphi(a\,;b\,;C) \textrm{ for every neighborhood } C \textrm{ of }\alpha}

\ceq{\hfill \langle b,\alpha\rangle\in B\smallsetminus{\EuScript D}}{\Rightarrow}{{\sim}\varphi(a\,;b\,;C) \textrm{ for every }C\textrm{ such that }\alpha\notin C}

\begin{fact}
  For every ${\EuScript D}\subseteq {\EuScript U}^{|z|}\times S^{\scriptscriptstyle|X|}$, the following are equivalent
  \begin{itemize}
    \item [1.] ${\EuScript D}$ is externally definable by some global $\varphi$-type
    \item [2.]  ${\EuScript D}$ is approximated by $\varphi(x\,;z\,;X)$.
  \end{itemize}
\end{fact}

\begin{proof}
  (1)$\Rightarrow$(2). Let $B\subseteq {\EuScript U}^{|z|}\times S^{\scriptscriptstyle|X|}$ be finite.
  For $\langle b,\alpha\rangle\in B\cap{\EuScript D}$ let $C_{\langle b,\alpha\rangle}$ be some neightborhood of $\alpha$. % such that $\varphi(x\,;b\,;C_{\langle b,\alpha\rangle})\in p$.
  Let $C$ be the union of all these neighborhoods.
  For $\langle b,\alpha\rangle\in B\smallsetminus{\EuScript D}$ let $\tilde C_{\langle b,\alpha\rangle}$ be some neightborhood of $\alpha$. % such that ${\sim}\varphi(x\,;b\,;\tilde C_{\langle b,\alpha\rangle})\in p$.
  Let $\tilde C$ be the union of all these neighborhoods.
  By the finiteness of $B$, we can require that $C$ and $\tilde C$ are disjoint. 
  Then (2) follows from the concistency of $p(x)$.

  (2)$\Rightarrow$(1). Note that the condition of approximability asserts the finite concistency of the type $p(x)$ that is union of the following two sets of formulas

\noindent\kern5ex $\Big\{\varphi(x\,;b\,;C)\ :\ \langle b,\alpha\rangle\in{\EuScript D}\textrm{ and }C\textrm{ neighborhood of }\alpha\Big\}$


\noindent\kern5ex $\Big\{{\sim}\varphi(x\,;b\,;C)\ :\ \langle b,\alpha\rangle\notin{\EuScript D}\textrm{ and }\alpha\notin C\Big\}$

The inclusion ${\EuScript D}\subseteq{\EuScript D}_p$ is immediate.
For the converse inclusion, let $\langle b,\alpha\rangle\notin{\EuScript D}$ and let $\tilde C$ be a neighborhood of $\alpha$ such that ${\sim}\varphi(x\,;b\,;\tilde C)$.
\end{proof}

  
\begin{proof}
  The theorem is an immediate consequence of the following three lemma.
\end{proof}

We say that ${\EuScript D}$ is approximable by $\varphi(x\,;z\,;C)$ if \emph{from below\/} if in Definition~\ref{def_approx} implication (2) holds for every $b\in{\EuScript U}^z$.

\begin{lemma}
  Under the assumptions of the theorem.
  If ${\EuScript D}$ is approximable from below then there are some $\langle a_i\ :\ i<k\rangle$ such that for every $b\in{\EuScript U}^z$\medskip

  \ceq{\hfill b\in{\EuScript D}}{\Leftrightarrow}{\bigvee_{i<k}\ \varphi(a_i\,;b\,;C)} 
\end{lemma}

\begin{proof}
We define a sequences $a_0,\dots,a_n$ recursively together with an auxiliary sequence $b_0,\dots,b_{n-1}$.
The elements in the sequence $a_0,\dots,a_n$ are choosen to obtain 

\ceq{1.\hfill\bigvee_{i=0}^n\ \varphi(a_i\,;b\,;C)}{\Rightarrow}{b\in{\EuScript D}}\hfill for every $b\in{\EuScript U}^z$

and

\ceq{2.\hfill b_i\in{\EuScript D}}{\Rightarrow}{\varphi(a_n\,;b_i\,;C)}\hfill for every $i<n$.\smallskip

This is possible because ${\EuScript D}$ is approximated from below.
Now choose $b_n$ and $C_n\cap C=\varnothing$ such that

\ceq{3.\hfill\bigwedge_{i=0}^n\ {\sim}\varphi(a_i\,;b_n\,;C_n)\ }{\text{and}}{b\in{\EuScript D}}%\smallskip

If such $b_n$ and $C_n$ do not exist, then 

\ceq{\hfill{\sim}\bigvee_{i=0}^n\ \varphi(a_i\,;b\,;\tilde C)\ }{\Rightarrow}{b\notin{\EuScript D}}\hfill for every $b\in{\EuScript U}^z$ and every $\tilde C\cap C=\varnothing$%\smallskip

which is equivalent to the converse of (1) and proves the lemma.
Suppose for a contradiction that the construction never ends.
Then, from (2) we otain $\varphi(a_n\,;b_i\,;C)$ for every $i<n$ and from (3) we obtain ${\sim}\varphi(a_i\,;b_n\,;C_n)$, for every $i\le n$.
This contradicts stability.
\end{proof}

\begin{lemma}
  Under the assumptions of the theorem.
  Let $m$ be maximal so that a sequence as in (2) of Fact~\ref{fact_stability_semicalssic} exists.
  Let $\bar x=\langle x_i\ :\ i<m\rangle$ where the $x_i$ are copies of $x$.
  Then the formula\medskip

  \ceq{\hfill\sigma(\bar x\,;z\,;X)}{=}{\bigwedge_{i<m}\ \varphi(x_i\,;z\,;X)}

  approximate ${\EuScript D}$ from below.
\end{lemma}

\begin{proof}
  Negate the claim and let $B$ witness that $\sigma(\bar x)$ does not approximate ${\EuScript D}$ from below.
  Suppose that $a_0,\dots,a_{n-1}$ and $b_0,\dots,b_{n-1}$ have been defined.
  Choose $a_n$ such that 

  \ceq{1.\hfill b\in {\EuScript D}}{\Rightarrow}{\varphi(a_n\,;b\,;C)}

  \ceq{2.\hfill{\sim}\varphi(a_n\,;b\,;\tilde C)}{\Rightarrow}{b\notin{\EuScript D}}\hfill for every $b\in B\cup\{b_0,\dots,b_{n-1}\}$ and some $\tilde C\cap C=\varnothing$.

  We write $C_i$ for the $\tilde C$ that witnesses (2) for $b_i$.
  
  Now, as the lemma is assume to fail, we can choose $b_n$ such that

  \ceq{3.\hfill\bigwedge_{i=0}^n\ \varphi(a_i\,;b_n\,;C)\ }{\text{and}}{b_n\notin{\EuScript D}.}

  We claim that the sequence we obtain contradicts stability.
  In fact, from (2) we obtain ${\sim}\varphi(a_n\,;b_i\,;C_i)$ for every $i<n$.
  On the other hand, by (1) we have that $\varphi(a_n\,;b_i\,;C)$ for every $i<n$.
\end{proof}

\begin{lemma}
  If $\varphi(x\,;z\,;X)$ is stable then $\sigma(\bar x\,;z\,;X)$ in the previous lemma is stable.
\end{lemma}

\begin{proof}
  Let $k$ such that every $m$-coloring of a graph of size $k$ has a monocromatic subgraph of size $>m$ 
  Let $\langle \bar a_i\,;b_i\,;C_i\ :\ i<k\rangle$ be a sequence witnessing instability.
  then for every $i<n<m$ either ${\sim}\varphi(a_{j,n}\,;b_i\,;C_i)$ for some $j<m$.
  By Ramsey theorem we can assume that there is a $j<m$ such that ${\sim}\varphi(a_{j,n}\,;b_i\,;C_i)$ obtains for $>m$ many $i$.
  Then we can extract a subsequence that contradicts the maximality of $m$.
\end{proof}

\end{comment}
\section{Attempt 2}

The proof of the theorem fails for stable but non-finitary stable formulas.
We can rescue part of the theorem we using a variant notion of approximation (hence loosing the correspondence to global types).

\begin{definition}\label{def_approx}\strut
  Let $C'$ be a neighboorhood of $C$.
  We say that ${\EuScript D}$ is \emph{$C'$-approximable\/} by $\varphi(x\,;z\,;C)$ if for every finite $B\subseteq {\EuScript U}^z$ there is an $a\in {\EuScript U}^x$ such that\smallskip

  \ceq{1.\hfill b\in {\EuScript D}}{\Rightarrow}{\varphi(a\,;b\,;C)}\quad and\smallskip

  \ceq{2.\hfill\varphi(a\,;b\,;C')}{\Rightarrow}{b\in {\EuScript D}}\hfill for every $b\in B$.\smallskip

  We say that the approximation is \emph{from below\/} if (2) holds for every $b\in{\EuScript U}^z$.
\end{definition}

% Note that latter implication may also be stated as follows\smallskip

% \ceq{3.\hfill b\notin{\EuScript D}}{\Rightarrow}{{\sim}\varphi(a\,;b\,;\tilde C)}\hfill for all $b\in B$ and some $\tilde C\cap C'=\varnothing$\smallskip

Clearly if ${\EuScript D}$ is $C'$-approximable for every neighborhood of $C$, then it is approximable.

\begin{lemma}\label{lem_stability_definability}
  Let $\varphi(x\,;z\,;C)$ be stable.
  Let $C'$ be a neighboorhood of $C$.
  Assme that ${\EuScript D}$ is $C'$-approximable by $\varphi(x\,;z\,;C)$.
  Then there are some $\langle a_{i,j}\ :\ i< k,\ j<m\rangle$ such that for every $b\in{\EuScript U}^z$

  \ceq{\hfill b\in{\EuScript D}}{\Rightarrow}{\bigvee_{i< k}\ \bigwedge_{j<m}\ \varphi(a_{i,j}\,;b\,;C')}\medskip

  \ceq{\hfill b\in{\EuScript D}}{\Leftarrow}{\bigvee_{i< k}\ \bigwedge_{j<m}\ \varphi(a_{i,j}\,;b\,;C)}
\end{lemma}

Note that for $\varphi(x\,;z\,;C)$ fixed, $k$ and $m$ depends on $C'$. 

\begin{proof}
  The theorem is an immediate consequence of the following three lemma.
\end{proof}

\begin{lemma}
  Under the assumptions of Lemma~\ref{lem_stability_definability}.
  If ${\EuScript D}$ is $C'$-approximable from below then there is a sequence $\langle a_i\ :\ i<k\rangle$ such that for every $b\in{\EuScript U}^z$\medskip

  \ceq{1.\hfill b\in{\EuScript D}}{\Rightarrow}{\bigvee_{i<k}\ \varphi(a_i\,;b\,;C')}\medskip 

  \ceq{2.\hfill b\in{\EuScript D}}{\Leftarrow}{\bigvee_{i<k}\ \varphi(a_i\,;b\,;C)} 
\end{lemma}

\begin{proof}
We define recursively the required parameters $a_i$ together with some auxiliary parameters $b_i$.
The element $a_n$ is choosen such that

\ceq{3.\hfill b\in{\EuScript D}}{\Leftarrow}{\varphi(a_n\,;b\,;C)}\hfill for every $b\in{\EuScript U}^z$

and

\ceq{4.\hfill b_i\in{\EuScript D}}{\Rightarrow}{\varphi(a_n\,;b_i\,;C)}\hfill for every $i<n$.\smallskip

This is possible because ${\EuScript D}$ is approximated from below.
Note that (3) immediately guarantees (2).
Now choose $b_n$ and $C_n\cap C'=\varnothing$ such that

\ceq{5.\hfill\bigwedge_{i=0}^n\ {\sim}\varphi(a_i\,;b_n\,;C_n)\ }{\text{and}}{b\in{\EuScript D}}%\smallskip

If such $b_n$ and $C_n$ do not exist, then 

\ceq{\hfill{\sim}\bigvee_{i=0}^n\ \varphi(a_i\,;b\,;\tilde C)\ }{\Rightarrow}{b\notin{\EuScript D}}\hfill for every $b\in{\EuScript U}^z$ and every $\tilde C\cap C'=\varnothing$%\smallskip

which is equivalent to (1) and proves the lemma.
Suppose for a contradiction that the construction never ends.
Then, from (4) we otain $\varphi(a_n\,;b_i\,;C)$ for every $i<n$ while from (5) we obtain ${\sim}\varphi(a_i\,;b_n\,;C_n)$, for every $i\le n$.
This contradicts stability.
\end{proof}

\begin{lemma}
  Under the assumptions of Lemma~\ref{lem_stability_definability}.
  Let $m$ be maximal so that a sequence as in (3) of Fact~\ref{fact_stability_semicalssic} exists~--~for $C'$ as given.
  Let $\bar x=\langle x_i\ :\ i\le m\rangle$ where the $x_i$ are copies of $x$.
  Then the formula\medskip

  \ceq{\hfill\sigma(\bar x\,;z\,;C)}{=}{\bigwedge_{i\le m}\ \varphi(x_i\,;z\,;C)}

  $C'$-weakly approximate ${\EuScript D}$ from below.
\end{lemma}

\begin{proof}
  Negate the claim and let $B$ witness that $\sigma(\bar x)$ does not approximate ${\EuScript D}$ from below.
  Suppose that $a_0,\dots,a_{n-1}$ and $b_0,\dots,b_{n-1}\notin{\EuScript D}$ have been defined.
  Choose $a_n$ such that for every $b\in B\cup\{b_0,\dots,b_{n-1}\}$

  \ceq{1.\hfill b\in {\EuScript D}}{\Rightarrow}{\varphi(a_n\,;b\,;C)}\quad and

  \ceq{2.\hfill\varphi(a_n\,;b\,;C')}{\Rightarrow}{b\in{\EuScript D}}
  
  Note that the latter implication is equivalent to 
  
  \ceq{3.\hfill b\notin{\EuScript D}}{\Rightarrow}{{\sim}\varphi(a_n\,;b\,;\tilde C)}\hfill for some $\tilde C\cap C'=\varnothing$.

  We write $C_i$ for the $\tilde C$ that witnesses (2) for $b_i$.
  Now, as the lemma is assume to fail, we can choose $b_n$ such that

  \ceq{\hfill\bigwedge_{i=0}^n\ \varphi(a_i\,;b_n\,;C)\ }{\text{and}}{b_n\notin{\EuScript D}.}

  We claim that the sequence has to stop after $\le m$ steps. 
  Otherwise we contradict the maximality of $m$.
  In fact, from (3) and $b_i\notin{\EuScript D}$ we obtain ${\sim}\varphi(a_n\,;b_i\,;C_i)$ for every $i<n$.
  On the other hand, by (1) we have that $\varphi(a_n\,;b_i\,;C)$ for every $i<n$.
\end{proof}

\begin{lemma}
  If $\varphi(x\,;z\,;C)$ is stable then $\sigma(\bar x\,;z\,;C)$ in the previous lemma is stable.
\end{lemma}

\begin{proof}
  Let $k$ be suffinciely large so that every $m$-coloring of a graph of size $k$ has a monocromatic subgraph of size $>m$.
  Let $\langle \bar a_i\,;b_i\,;C_i\ :\ i<k\rangle$ be a sequence witnessing instability as in (3) of Fact~\ref{fact_instable3}.
  Then for every pair $i<n$ there is some $j<m$ such that ${\sim}\varphi(a_{j,n}\,;b_i\,;C_i)$.
  By the choice of $k$ there is a $j<m$ such that ${\sim}\varphi(a_{j,n}\,;b_i\,;C_i)$ obtains for $>m$ many $i$.
  Then we can extract a subsequence that contradicts the maximality of $m$.
\end{proof}

\begin{theorem}[ (conjecture)]\label{thm_stable_definable}
  Let $\varphi(x\,;z\,;C)$ be stable.
  Let ${\EuScript D}$ be approximable by $\varphi(x\,;z\,;C)$.
  Then there is $\pi(z\,;C)$, a small set of formulas of the form $\varphi(a\,;z\,;C)$, with $a\in{\EuScript U}^x$, such that

  \ceq{\hfill b\in{\EuScript D}}{\Leftrightarrow}{\pi(b\,;C)}\hfill for every $b\in{\EuScript U}^z$.
\end{theorem}




%%%%%%%%%%%%%%%%%%%%%%%%%
%%%%%%%%%%%%%%%%%%%%%%%%%%
%%%%%%%%%%%%%%%%%%%%%%%%%%
%%%%%%%%%%%%%%%%%%%%%%%%%%
%%%%%%%%%%%%%%%%%%%%%%%%%%
\section{Attempt 2}
\def\medrel#1{\parbox{5ex}{\hfil $#1$}}
\def\ceq#1#2#3{\parbox[t]{22ex}{$\displaystyle #1$}\medrel{#2}{$\displaystyle #3$}}

The proof of the main theorem requires the following stronger notion of approximation.

\begin{definition}\label{def_e_approx_X}\strut
  We say that ${\EuScript D}$ is \emph{$\varepsilon$-approximable\/} by $\varphi(x\,;z\,;X)$ if for every finite $B\subseteq{\EuScript U}^z$ there is an $a\in{\EuScript U}^x$ such that\smallskip

  \ceq{1.\hfill \langle b,C\rangle\in{\EuScript D}}{\Rightarrow}{\varphi(a\,;b\,;C)}\quad and\smallskip

  \ceq{2.\hfill\langle b,C\rangle\in {\EuScript D}}{\Leftarrow}{\varphi(a\,;b\,;C^\varepsilon)}\hfill for every $b\in B$ and every $C$.\smallskip
\end{definition}


\begin{theorem}\label{thm_epsilon_stability_definability2}
  Let $\varphi(x\,;z\,;X)$ be stable.
  Assume that ${\EuScript D}$ is $\varepsilon$-approximable by $\varphi(x\,;z\,;X)$.
  Then there are some $\langle a_{i,j}\ :\ i< k,\ j<m\rangle$ such that for every $b\in{\EuScript U}^z$ and every $C$\medskip

  \ceq{\hfill \langle b,C\rangle\in{\EuScript D}}{\Rightarrow}{\bigvee_{i< k}\ \bigwedge_{j<m}\ \varphi(a_{i,j}\,;b\,;C^\varepsilon)}\medskip

  \ceq{\hfill \langle b,C\rangle\in{\EuScript D}}{\Leftarrow}{\bigvee_{i< k}\ \bigwedge_{j<m}\ \varphi(a_{i,j}\,;b\,;C)}
\end{theorem}

\begin{proof}
  The theorem is an immediate consequence of the following three lemmas.
\end{proof}

\begin{lemma}
  Let $\varphi(x\,;z\,;X)$ be $\varepsilon$-stable.
  Assume that ${\EuScript D}$ is approximable by $\varphi(x\,;z\,;X)$ from below.
  Then there is are some $\langle a_i\ :\ i<k\rangle$ such that for every $b\in{\EuScript U}^z$ and every $C$\medskip

  \ceq{1.\hfill \langle b,C\rangle\in{\EuScript D}}{\Rightarrow}{\bigvee_{i<k}\ \varphi(a_i\,;b\,;C^\varepsilon)}\medskip 

  \ceq{2.\hfill \langle b,C\rangle\in{\EuScript D}}{\Leftarrow}{\bigvee_{i<k}\ \varphi(a_i\,;b\,;C)} 
\end{lemma}

\begin{proof}
Let $\varepsilon$ be given
  We define recursively the required parameters $a_i$ together with some auxiliary parameters $b_i$.
The element $a_n$ is choosen such that

\ceq{3.\hfill \langle b,C\rangle\in{\EuScript D}}{\Leftarrow}{\varphi(a_n\,;b\,;C)}\hfill for every $b\in{\EuScript U}^z$ and every $C$

and

\ceq{4.\hfill \langle b_i,C\rangle\in{\EuScript D}}{\Rightarrow}{\varphi(a_n\,;b_i\,;C)}\hfill for every $i<n$ and every $C$.\smallskip

This is possible because ${\EuScript D}$ is approximated from below.
Note that (3) immediately guarantees (2).
Now, assume (1) fails for $k=n$, and choose $b_n$ and $C_n$ witnessing this.
Then, by Fact~\ref{fact_otto}, for some $\tilde C_n\cap C_n^\varepsilon=\varnothing$

\ceq{5.\hfill \langle b_n,C_n\rangle\in{\EuScript D}}{\&}{\bigwedge_{i=0}^n\ {\sim}\varphi(a_i\,;b_n\,;\tilde C_n)}%\smallskip

Suppose for a contradiction that the construction never ends.
Then, from (4) we otain $\varphi(a_n\,;b_i\,;C_n)$ for every $i<n$ while from (5) we obtain ${\sim}\varphi(a_i\,;b_n\,;\tilde C_n)$, for every $i\le n$.
This contradicts stability.
\end{proof}

\begin{lemma}
  Let $\varphi(x\,;z\,;X)$ be stable.
  Assume that ${\EuScript D}$ is $\varepsilon$-approximable by $\varphi(x\,;z\,;X)$.
  Let $m$ be maximal so that a sequence as in (3) of Theorem~\ref{thm_epsilon_stable} exists.
  Let $\bar x=\langle x_i\ :\ i\le m\rangle$ where the $x_i$ are copies of $x$.
  Then the formula\smallskip

  \ceq{\hfill\sigma(\bar x\,;z\,;X)}{=}{\bigwedge_{i\le m}\ \varphi(x_i\,;z\,;X)}\smallskip

  approximate ${\EuScript D}$ from below.
\end{lemma}

\begin{proof}
  Negate the claim and let $B$ witness that $\sigma(\bar x)$ does not approximate ${\EuScript D}$ from below.
  Suppose that $a_0,\dots,a_{n-1}$ and $b_0,\dots,b_{n-1}\notin{\EuScript D}$ have been defined.
  Choose $a_n$ such that for every $b\in B\cup\{b_0,\dots,b_{n-1}\}$ and every $C$

  \ceq{1.\hfill\langle b,C\rangle\in{\EuScript D}}{\Rightarrow}{\varphi(a_n\,;b\,;C)}\quad and

  \ceq{2.\hfill\langle b,C\rangle\in{\EuScript D}}{\Leftarrow}{\varphi(a_n\,;b\,;C^\varepsilon)}
  
  Note that the latter implication is equivalent to: for every $C$ there is some  $\tilde C\cap C^\varepsilon=\varnothing$ such that 
  
  \ceq{3.\hfill \langle b,C\rangle\notin{\EuScript D}}{\Rightarrow}{{\sim}\varphi(a_n\,;b\,;\tilde C)}.%\hfill  for every $b\in B\cup\{b_0,\dots,b_{n-1}\}$.

  Now, as the lemma is assumed to fail, we can choose $b_n$ and $C_n$ such that

  \ceq{4.\hfill \langle b_n,C_n\rangle\notin{\EuScript D}}{\&}{\bigwedge_{i=0}^n\ \varphi(a_i\,;b_n\,;C_n)}

  Note that (4) ensure that $\langle b_i,C_i\rangle\notin{\EuScript D}$ for every $i$.
  Then there is some $\tilde C_i$ that witnesses (3) for $\langle b_i,C_i\rangle\notin{\EuScript D}$.
  We claim that the procedure has to stop after $\le m$ steps. 
  In fact, from (3) we obtain ${\sim}\varphi(a_n\,;b_i\,;\tilde C_i)$ for every $i<n$.
  On the other hand, by (4) we have that $\varphi(a_i\,;b_n\,;C_n)$ for every $i<n$.
  Therefore, $\langle a_{m-i}\,;b_{m-i}\,;C_{m-i}\,;\tilde C_{m-i}\ :\ i\le m\rangle$ contradicts the maximality of $m$.
\end{proof}

\begin{lemma}
  If $\varphi(x\,;z\,;C)$ is stable then $\sigma(\bar x\,;z\,;C)$ in the previous lemma is $\varepsilon$-stable.
\end{lemma}

\begin{proof}
  Let $\varepsilon$ be given and let $m$ be maximal such that a sequence as in (3) of Theorem~\ref{thm_epsilon_stable} exists.
  Let $k$ be suffinciely large so that every $m$-coloring of a graph of size $k$ has a monocromatic subgraph of size $>m$.
  Let $\langle \bar a_i\,;b_i\,;C_i\,;\tilde C_i\ :\ i<k\rangle$ be a sequence witnessing instability as in (3) of Theorem~\ref{thm_epsilon_stable}.
  Then for every pair $i<n$ there is some $j<m$ such that ${\sim}\varphi(a_{j,n}\,;b_i\,;\tilde C_i)$.
  By the choice of $k$ there is a $j<m$ such that ${\sim}\varphi(a_{j,n}\,;b_i\,;\tilde C_i)$ obtains for $>m$ many $i$.
  Then we can extract a subsequence that contradicts the maximality of $m$.
\end{proof}

\begin{lemma}
  Let $\varphi(x\,;z\,;X)$ be $\varepsilon$-stable.
  Assume that ${\EuScript D}$ is $\delta$-approximable by $\varphi(x\,;z\,;X)$.
  Let $m$ be maximal so that a sequence as in (3) of Theorem~\ref{thm_epsilon_stable} exists.
  Let $\bar x=\langle x_i\ :\ i\le m\rangle$ where the $x_i$ are copies of $x$.
  Then the formula\smallskip

  \ceq{\hfill\sigma(\bar x\,;z\,;X)}{=}{\bigwedge_{i\le m}\ \varphi(x_i\,;z\,;X)}\smallskip

  approximate ${\EuScript D}$ from $\varepsilon$-below.
\end{lemma}

\begin{proof}
  Negate the claim and let $B$ witness that $\sigma(\bar x)$ does not approximate ${\EuScript D}$ from $\varepsilon$-below.
  Suppose that $a_0,\dots,a_{n-1}$ and $b_0,\dots,b_{n-1}\notin{\EuScript D}$ have been defined.
  Choose $a_n$ such that for every $b\in B\cup\{b_0,\dots,b_{n-1}\}$ and every $C$

  \ceq{1.\hfill\langle b,C\rangle\in{\EuScript D}}{\Rightarrow}{\varphi(a_n\,;b\,;C)}\quad and

  \ceq{2.\hfill\langle b,C^\varepsilon\rangle\in{\EuScript D}}{\Leftarrow}{\varphi(a_n\,;b\,;C^\varepsilon)}
  
  Note that the latter implication is equivalent to: for every $C$ there is some  $\tilde C\cap C^\varepsilon=\varnothing$ such that 
  
  \ceq{3.\hfill \langle b,C^\varepsilon\rangle\notin{\EuScript D}}{\Rightarrow}{{\sim}\varphi(a_n\,;b\,;\tilde C)}.%\hfill  for every $b\in B\cup\{b_0,\dots,b_{n-1}\}$.

  Now, as the lemma is assumed to fail, we can choose $b_n$ and $C_n$ such that

  \ceq{4.\hfill \langle b_n,C_n^\varepsilon\rangle\notin{\EuScript D}}{\&}{\bigwedge_{i=0}^n\ \varphi(a_i\,;b_n\,;C_n)}

  Note that (4) ensure that $\langle b_i,C_i^\varepsilon\rangle\notin{\EuScript D}$ for every $i$.
  The there is some $\tilde C_i$ that witnesses (3) for $\langle b_i,C_i^\varepsilon\rangle\notin{\EuScript D}$.
  We claim that the procedure has to stop after $\le m$ steps. 
  In fact, from (3) we obtain ${\sim}\varphi(a_n\,;b_i\,;\tilde C_i)$ for every $i<n$.
  On the other hand, by (4) we have that $\varphi(a_i\,;b_n\,;C_n)$ for every $i<n$.
  Therefore, $\langle a_{m-i}\,;b_{m-i}\,;C_{m-i}\,;\tilde C_{m-i}\ :\ i\le m\rangle$ contradicts the maximality of $m$.
\end{proof}
